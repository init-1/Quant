\section{Build A New Factor\label{sec:steps}}

In this section, we take \emph{Book to Price Ratio} as an example to show how to construct a new factor
and add it to our (existing) factor library.

Book to price ratio is a popular multiple to value stocks.
It is defined as
\begin{equation}\label{eq:BKTOPRICE}
  \text{Book to Price} = \frac{\text{Total Equity Reported}}{\text{Price} \times\text{Shares Outstanding}}
\end{equation}
A higher B/P ratio could mean that the stock is undervalued,
or, the higher the values of the factor, the better the associated stock.

Now follow the steps discussed below (subsection (\ref{sec:SetPath})-(\ref{sec:Populate}) to create this factor.


\subsection{Set Path\label{sec:SetPath}}

 Set the path so that \matlab{} can find our toolbox:
\begin{lstlisting}
  basepath = ['Y:\' getenv('USERNAME') '\QuantStrategy\Analytics\'];
  addpath([basepath 'Utility']);            % for xts, myfints
  addpath([basepath 'date']);               % for DateBasis and Freq
  addpath([basepath 'DB']);                 % DataBase related stuff
  addpath([basepath 'myfintsUtility']);     % for bunch of things
  addpath([basepath 'FactorLib\updated']);  % all up-to-date factors code
\end{lstlisting}

These pathes is not only for creating factors, 
but also for developing quantitative models.

\subsection{Create Class File\label{sec:CreateClassFile}}

Every factor is defined as a class and organized in a class file under the folder
\texttt{\$/QuantStrategy/\allowbreak{}Analytics/\allowbreak{}FactorLib/\allowbreak{}Dev})
where \$ usually is you home folder.
The class file should have the same name as the class name 
(in current case, it is \texttt{BKTOPRICE}) and be structured as
\begin{lstlisting}
  classdef BKTOPRICE < FacBase
      methods (Access = protected)
          function factorTS = build(o, secIds, startDate, endDate)
          ...
          end
		
          function factorTS = buildLive(o, secIds, runDate)
          % ONLY needed if live version largely different from nonlive (build) version.
          ...
          end
      end
  end
\end{lstlisting}
       Every factor class must include method \mcode{build}, which takes four parameters:
       \footnote{These parameters will be passed by \mcode{create} method in \mcode{FacBase}
       which unified for all factor classes.}
       \begin{argdesc}
          \item [o] (lowercase letter) the object of the class.
                \footnote{In most cases and as sort of convention, we use \mcode{o} to represent the object inside a class file.}
                Bearing in mind when program running here, aside members derived from \myfints{},
                the following members already be set and can be accessed:
                \begin{itemize}
                   \item \mcode{DateBasis} keeps the frequency used internally of data items 
                         and details related to frequency conversion. 
                         See the subsection \emph{Understand Frequency} for detailed explanation.
                         \emph{You can \emph{change} this member if necessary, 
                         but should do this before loading any data items.}
                   \item \mcode{isLive} is a \emph{read only} member 
                         indicating if we are running in live or back-test mode.
                   \item \mcode{freq}, also \emph{read only}, 
                         is the frequency of the factor returned to users.
                         When returning the final calculated factor to users (by \mcode{create} of \mcode{FacBase}) , 
                         \mcode{FacBase} will re-sample the factor to this frequency.
                         This member provides a way to let your code to know the final frequency 
                         and is mainly controlled by \mcode{FacBase}, 
                         you should not change its value (though we do not impose this restriction).
                \end{itemize}

          \item [secIds] a cell vector of strings representing names of securities (i.e., stocks).
                The factor will be calculated for these stocks.

          \item [startDate] start date of the period in which the factor is calculated.
          \item [endDate] end date of the period in which the factor is calculated.
       \end{argdesc}
       As said in the comments of the code, 
       you should \emph{only} include the live version (\mcode{buildLive}) in the class
       if it is significantly different from the nonlive version (\mcode{build}).
       If they are only marginally different, 
       you can branch your code by checking the value of \mcode{o.isLive}.

       Note that \mcode{buildLive} takes few parameters 
       because those left out are meaningless in live.

\nopagebreak
\addcontentsline{toc}{subsubsection}{Understand Frequency}
\tikzset{
  normal border/.style={orange!30!black!10, decorate, 
     decoration={random steps, segment length=2.5cm, amplitude=.7mm}},
  torn border/.style={orange!30!black!5, decorate, 
     decoration={random steps, segment length=.5cm, amplitude=1.7mm}}}

\begin{parchment}[Understand Frequency]
       \small\sffamily
       Two frequencies are used inside a factor class. 
       The first, accessing via \mcode{o.freq} where \mcode{o} is the factor object,
       is the frequency of the final factor constructed by the factor class.
       It is specified as second parameter when users create a factor by calling
       (\mcode{create} can be passed more parameters in name-value pairs following \mcode{endDate}.)
\begin{lstlisting}[numbers=none]
  fac = create(BKTOPRC, 'M', isLive, secIds, startDate, endDate);
\end{lstlisting}
       where \mcode{'M'} stands for monthly frequency.
       The factor returned in \mcode{fac} then is monthly sampled.
       We thereafter refer it as \emph{target frequency}.

       The second, almost unnoticeable but equally important, stored as
       a \mcode{DataBasis} object inside the factor,
       is the frequency of data items involved in calculating the factor value.
       We refer this frequency as \emph{calc frequency}
       This frequency can be the same as target frequency.
       But at least should be higher than the target frequency.
       The reason is obvious:
       when the factor returned to user, \mcode{FacBase}
       will convert the factor in calc frequency to the target frequency,
       and if the calc frequency is more sparser than the target frequency,
       information is going to be lost.

       Then when and why do we need the calc frequency higher than (instead of equal to) the target frequency?
       Suppose we need a factor both weekly and monthly sampled,
       it seems we should pick up the weekly calc frequency.
       This is fine for weekly target frequency, 
       but for monthly target frequency,
       converting from weekly to monthly is inaccurate, 
       since one month has 30/7 weeks, 
       and it is not an integer.
       So a better frequency in this case should be daily.

       Choice of calc frequency also depends on the definition of the factor.
       Consider a factor of 5 day momentum of prices,
       clearly its calc frequency should be set to daily.
       To force a factor to use a fixed calc frequency 
       (instead of calc frequency passed by \mcode{create}),
       you can set its \mcode{dateBasis} member to that frequency
       before loading any items:
\begin{lstlisting}[numbers=none]
    o.dateBasis = DateBasis('BD');
\end{lstlisting}
       where \mcode{'BD'} indicates business days.

       \mcode{DateBasis} is a class in charge of frequency conventions.
       Aside from \mcode{'BD'}, we have other mnemonics:
          \hspace*{1cm}\begin{tabular}{ll}
            \mcode{'BD'} & business daily \\
            \mcode{'BW'} & business weekly \\
            \mcode{'BM'} & business monthly \\
            \mcode{'D'}  & calendar daily \\
            \mcode{'W'}  & calendar weekly \\
            \mcode{'M'}  & calendar monthly \\
         \end{tabular}

      \vspace{.5cm}You can also define your own calc frequency by
\begin{lstlisting}[numbers=none]
    o.dateBasis = DateBasis(freqBasis, nY, nQ, nM, nW, nD, isBusDay);
\end{lstlisting}
      where 
      \begin{argdesc}
         \item [freqBasis] a char of \mcode{'D'}, \mcode{'W'}, \mcode{'M'}, \mcode{'Q'} or \mcode{'A'}, 
               standing for daily, weekly, monthly, quarterly and annually, respectively.
               Frequency indicators are interpreted by class \mcode{Freq}.
         \item [nY, nQ, nM, nW, nD] are all integers, 
               indicating how many periods are contained in a year, a quarter, a month, a week and a day,
               respectively.
         \item [isBusDay] indicates if only business days are accounted;
               or, equivalently, if whe should exclude holidays.
               \mcode{isbusDay} is processed by class \mcode{Freq}.
      \end{argdesc}
      For example, the \mcode{DateBasis('BD')} previously actually is defined as:
\begin{lstlisting}[numbers=none]
    o.dateBasis = DateBasis('D', 252, 21*3, 21, 5, 1, true);
\end{lstlisting}
      where we abide by the convention that one year has 252 business days, and one month has 21 business days.
\end{parchment}

\subsection{Write Implementation Code\label{sec:Implementation}}

Since \mcode{FacBase} does most of routine job,
you only need to focus on the definition part of the factor.
Taking a look at the \mcode{BKTOPRICE} and 
comparing it with equation (\ref{eq:BKTOPRICE}) on page \pageref{eq:BKTOPRICE}
could give you some sense about this point:
\begin{lstlisting}
classdef BKTOPRICE < FacBase
    methods (Access = protected)
        function factorTS = build(o, secIds, startDate, endDate)
            % §\textcolor{blue}{\textbf{1}}§ Load data items
            bookValue = o.loadItem(secIds,'D000686133',startDate,endDate,4);
            closePrice = o.loadItem(secIds,'D001410415',startDate,endDate);
            shares = o.loadItem(secIds,'D001410472',startDate,endDate);
            
            % §\textcolor{blue}{\textbf{2.1}}§ Calculate the mean of last n-quarter book value for each point in time
            bookValueMean = ftsnanmean(bookValue{:});
            bookValueMean = backfill(bookValueMean, o.DCF('3M'),'entry');
            
            % §\textcolor{blue}{\textbf{2.2}}§ Here the equation §\ref{eq:BKTOPRICE}§
            factorTS = (1000000*bookValueMean./shares)./closePrice;
        end
    end
end
\end{lstlisting}

Basically, the implementation follows a regular procedure:
\begin{enumerate}
   \item Load data items using \mcode{o.loadItem}.
         \footnote{Note not all data items can be loaded this way, see explanation below.}
         If you loading multiple items using this function
         with the same parameters of \mcode{secIds}, \mcode{startDate} and \mcode{endDate},
         then the items you finally get are already aligned along both the time and the field dimension.
   \item Plug items into definition equation(s).
         Some equations may require a bit more complex logic and more code than a few lines,
         sometimes even need to write some help functions.
         So this is the key part you should focus on.
   \item Do any necessary processing of the calculated factor, 
         like backfilling
         (though backfilling is not necessary in most cases, 
         since \mcode{o.loadItem} will do backfilling according to the natural frequency of the raw data items. 
         Also note that \mcode{FacBase} also set all \mcode{inf}s to \mcode{NaN}s.
\end{enumerate}

This pattern is not mandatory. 
Based on factors, you may employ different ways to implement factors.
However, always keep in mind that item aligning, item backfilling,
re-sampling the final factor to target frequency,
and others mentioned above have been done by the framework.

Sometimes a group of factors may share the same structure, 
just a few things different.
In this case, it is better to abstract a base class (derived from \mcode{FacBase}, of course)
which do the common work,
and then concrete factor classes, each do their own unique things.

If you have decided to add a live version of build (\mcode{buildLive}),
follow the same guild line.
But even you do not have one,
the framework will provide it for you
(defined in \mcode{FacBase} and inherited by your own factor classes),
and 
\begin{lstlisting}[numbers=none]
   o.buildLive(secIds, runDate)      % o.isLive == false
\end{lstlisting}
is equivalent to
\begin{lstlisting}[numbers=none]
   o.build(secIds, runDate, runDate) % o.isLive == true, same start date and end date
\end{lstlisting}
bearing in mind that \mcode{isLive} has different values corresponding to live and back-test cases.

%            A direct simply way is to remove the class restrictions. 
%            This can be done by modifying the class definition file. 
%            For our \mcode{BKTOPRICE} example, the modified class definition file for debugging
%            looks like
%%%%%%
%\begin{lstlisting}
%classdef Book2Price §\sout{\textcolor{red}{< FacBase}}§
%    methods (§\sout{\textcolor{red}{Access = protected, }}§Static)
%        factorTS = build(secIds, startDate, endDate, targetFreq)
%        factorTS = buildLive(secIds, endDate)
%    end
%end
%\end{lstlisting} 
%%%%%%%
% 	          Note the \sout{\textcolor{red}{crossed red}} part is removed temporarily.
%	          Then you can call your function like this:\\
%	             \hspace*{.5cm} \mcode{fts = Book2Price.build('', '1996-01-01', '2011-03-31', 'M');}\\
%	          Note that you \texttt{@Book2Price} folder must be either in \matlab{} path set or directly
%	          under current work folder.
% 
  
\subsection{Test\label{sec:Test}}

Test you code by calling \mcode{create}:
\begin{lstlisting}
% When isLive == false, create actually call build
  fac = create(factor_object, isLive, targetFreq, secIds, startDate, endDate, 'name', val,...)
% When isLive == true, create actually call buildLive
  fac = create(factor_object, isLive, secIds, runDate, 'name', val,...)
\end{lstlisting}

    Parameters \mcode{secIds}, \mcode{startDate}, \mcode{endDate} and \mcode{runDate} 
    have the same meanings as in \mcode{build} and \mcode{buildLive} explained in last step
    (subsection \ref{sec:Implementation}).
    
    The first parameter, \mcode{factor_object}, 
    is an (usually empty) factor object that can be created either by direct putting the class (default) constructor there:
    \footnote{Because we never implement a constructor for factor classes, 
             \matlab{} produces a default constructor.
             Also note that constructor has the same name as the class.}   
\begin{lstlisting}[numbers=none]
  fac = create(BKTOPRICE, ...)
\end{lstlisting}
    or through a function handle to the class constructor
\begin{lstlisting}[numbers=none]
  fac = create(fun_handle(), ...)  % fun_handle should be @BKTOPRICE
\end{lstlisting}
    The pair of parentheses after the function handle is must, 
    otherwise \matlab{} thinks you are passing a function handle instead of the factor object.

   The second parameter, \mcode{isLive},
   decides what parameters should be followed.
   If it is \mcode{true}, it tell \mcode{create} to create the live version of
   the factor and can only be called as shown in line 4 in the above code snippet;
   if it is \mcode{false}, back-test version of factor is going to be created
   as shown in line 2 in the same code snippet.

   From third parameter on, the two versions of build diverge.
   For \mcode{build}, it is target frequency.
   \mcode{buildLive} does not need that parameter since it only need to return values for the \mcode{runDate}.

   The \mcode{'name'}-\mcode{val} pairs are used to provide additional information.
   \mcode{'name'} can be of the following:

   \vspace*{.2cm}\begin{tabular}{r>{\sffamily}p{11cm}<{}}
        \mcode{'id'}   & factor id \\
        \mcode{'name'} & factor name \\
        \mcode{'type'} & factor type \\
        \mcode{'higherTheBetter'} & true if higher the factor value, the better the factor; false the other way around \\
        \mcode{'dateBasis'} & for calc frequency, val passed should be type of \mcode{DateBasis} \\
   \end{tabular}

   \vspace{.3cm}
   Usually, \mcode{'id'}, \mcode{'name'}, \mcode{'type'} and \mcode{'higherTheBetter'} 
   are values from factor registering database (if the factor has been registered).
   For an example of using these \mcode{'name'}-\mcode{val} parameters,
   see \texttt{LoadFactorTS.m} which not only load factor values, but also its associated
   registered information.
   For how to register a factor, go to next step.

   Finally note that \mcode{build} and \mcode{buildLive} can not be called directly
   since factor creation need \mcode{create} to do the common routine jobs 
   while \mcode{build} to deal with the factor definition.

\subsection{Register Factor\label{sec:Register}}

Having a factor been done in coding, debugging and testing, 
you should register it into the database, 
so that it can be known by and used in models.
Here is the format:
\begin{lstlisting}
id = Factory.Register2DB(...
      'Book to Price Ratio' ...                             % name
    , 'Book value per share divided by price per share' ... % description
    , 'BKTOPRICE' ...                                       % name of factor class
    , true ...                                              % is it the higher the better?
    , true ...                                              % is it active currently?
    , true);                                                % can it be used in production?
\end{lstlisting}
The \mcode{id} returned is an unique string of form like \mcode{'F00172'} and
can be referenced when- and wherever it is needed,
like in \texttt{LoadFactorTS} when you load populated factors,
in factor combination (blending) where you need to specify factors to be combined, etc.

The database table for keeping factor registering information is
\texttt{quantstrategy\allowbreak{}.fac\allowbreak{}.factormstr}.
	
\subsection{Populate\label{sec:Populate}}

When needing a factor, you can call \texttt{LoadFactorTS} providing it with factor id and other information.
\texttt{LoadFactorTS} can either calculate the factor on the fly or
read it from database where pre-calculated values of factors
has been populated.
There are pros and cons for both ways:
The benefits of on-the-fly is obvious: 
it can always reflect newest changes in data items and easy to maintain the consistence of factors,
not to mention it also reduces data redundancy.
Equally obvious is its drawback: it is slow in our current database infrastructure.

Populating factor is on the opposite side of on-the-fly.
The class \mcode{Factory} is used to do the populating and
if needed, please refer to the demo code in
\texttt{run\allowbreak{}Registered.m} and \texttt{run\allowbreak{}Registered\allowbreak{}Live.m}.

\subsection{More Examples}

For helping you get more sense about creating factors,
we present two more examples in this subsection,
each of which shows different aspects of factor creation.

\nopagebreak
\addcontentsline{toc}{subsubsection}{5-Day Price Reversal (REVSAL5D)}
\begin{parchment}[5-Day Price Reversal (REVSAL5D)]
\small\sffamily

Price reversal is a well-know short-term anomaly.  
The expectation is that stocks that have performed very well in the last 5 days prior to stock selection days 
will underperform over the next month and vice versa.
Its formula is 
\[
    \text{5-Day Price Reversal} = \text{\textit{Z}-Score}_{\text{cross sectional}}
       \left(\frac{\text{Price}_{\text{current}}}{\text{Price}_{\text{5 days lag}}}-1\right).
\]
Support we want monthly factors (i.e., target frequency is \mcode{'M'}),
what calc frequency should you choose?
Since the factor itself is defined based on daily,
the calc frequency definitely shoud be daily.
In this case, based on the nature
the factor should decide its calc frequency instead of
setting outside by \mcode{create} 
(Remember the \mcode{'name'}-\mcode{val} parameter can be used
to pass a \mcode{'dateBasis'}).
That is the first thing to do in the code:
changing the \mcode{dateBasis} to daily
(actually, business daily, since price only quoted in business day)
no matter what it is set outside by \mcode{create} (and therefore by the framework).

\begin{lstlisting}
classdef REVSAL5D < FacBase
    methods (Access = protected)
        function factorTS = build(o, secIds, startDate, endDate)
            o.dateBasis = DateBasis('BD'); % force the calc frequency to be 'BD'
            
            sDate = datestr(addtodate(datenum(startDate),-1,'M'),'yyyy-mm-dd');
            closePrice = o.loadItem(secIds,'D001410415',sDate,endDate);
            closePrice = backfill(closePrice,o.DCF('3D'),'entry');
            factorTS = Price2Return(closePrice,5);
        end
    end
end
\end{lstlisting}

Note also how we use \mcode{DCF} function to calculate a given tenor.
A tenor is a period or duration represented by something like \mcode{'5D'} (5 days),
\mcode{'2M'} (2 months), \mcode{'6W'} (6 weeks), etc.
The units can be used in a tenor representation are the same
as frequency units (i.e., \mcode{'D'}, \mcode{'W'}, \mcode{'M'}, \mcode{'Q'}, \mcode{'Y'}).
\mcode{DCF}, standing for \emph{Date Conversion Factor},
converts the tenor given in the first parameter to an integer
number based on the calc frequency;
the number is the number of basic period represented by calc frequency;
if it is omitted, it is assumed to be 1.
For example, if calc frequency is \mcode{'BD'},
then \mcode{o.DCF('1M')} equals to 21, meaning in one month there are 21 business days.
In current example, since we force the calc frequency to be \mcode{'BD'},
\mcode{o.DCF('3D')} in line 8 is 3.
Since the conversion is based on calc frequency represented by \mcode{dateBasis}
of the factor object \mcode{o}, the items involved (\mcode{closePrice})
must also be obtained by methods in \mcode{FacBase};
to be specific, methods of \mcode{loadItem} and \mcode{loadBondYield} for now.

For items (\myfints{} objects) obtained by other ways,
you can set its frequency to be the same as calc frequency; for example:
\begin{lstlisting}[numbers=none]
   gics = LoadQSSecTS(secIds, 913, 0, startDate, endDate, o.dateBasis.freqBasis);
\end{lstlisting}
then you can still use \mcode{DCF} as before:
\begin{lstlisting}[numbers=none]
   gics = lagts(gics, o.DCF('1M'));
   gics = backfill(gics, o.DCF('1M');
\end{lstlisting}

Finally, we mention here that 
\mcode{FacBase} provides shortcuts for \mcode{lagts} and \mcode{leadts}:
\begin{lstlisting}[numbers=none]
  fts = o.lagfts(fts, '3M')   <==> fts = lagts(fts, o.DCF('3M');
  fts = o.leadfts(fts, '3M')  <==> fts = leadts(fts, o.DCF('3M');
\end{lstlisting}

\end{parchment}


\nopagebreak
\addcontentsline{toc}{subsubsection}{Blended EPS Yield (BEPSY)}
\begin{parchment}[Blended EPS Yield (BEPSY)]
\small\sffamily

Rather than behaving rationally as implied in standard financial theory, 
investors tend to make systematic cognitive errors which a truly objective investor can exploit. 
Such errors include overreacting to bad news, 
confusing a bad company with a bad stock, 
and assuming poorly performing stocks will continue to behave badly.  
Each of these can result in an inappropriately 
low stock price which can lead to investment opportunity.

The formula of the factor is therefore defined as
\[
    \text{Blended EPS Yield} = \frac{\text{Blended Stock EPS}}{\text{Stock Price}}
\]
where blended stock EPS is average of the actual EPS and estimate EPS if both are available,
else it is the one available if only one of them is available,
otherwise, blended stock EPS is set to null

Here's the code:
\begin{lstlisting}
classdef BEPSY < FacBase
    methods (Access = protected)
        function factorTS = build(o, secIds, startDate, endDate)
            % set the sDate 3 month earlier than the input startDate to ensure some look back
            % for the first observation
            sDate = datestr(addtodate(datenum(startDate),-3,'M'),'yyyy-mm-dd');
            EPS_Act = o.loadItem(secIds,'D000432130',sDate,endDate);
            if o.isLive
                EPS_FY1 = o.loadItem(secIds,'D000448965',sDate,endDate); % QFS FY1 EPS mean
            else            
                EPS_FY1 = o.loadItem(secIds,'D000435584',sDate,endDate);
            end
            closePrice = o.loadItem(secIds,'D001410415',sDate,endDate);
            
            % match the items
            EPS_Act = backfill(EPS_Act, '2M', 'entry');
            EPS_FY1 = backfill(EPS_FY1, '2M', 'entry');
            
            % calculate factor value
            EPS_Blended = ftsnanmean(EPS_Act, EPS_FY1);
            factorTS = EPS_Blended ./ closePrice;
        end
    end
end
\end{lstlisting}
Note that all income statement items are annualized by summing up last four quarter values. 
But the most notible thing is how we use \mcode{o.isLive} to differentiate live and back-test version of
build in a single \mcode{build} function
instead of having them seperately in \mcode{build} and \mcode{buildLive}.

Finally, we emphasis here that unlike last example you can replace \mcode{o.DCF('3D')} by \mcode{3}
since forcing of calc frequency,
here you must stick to \mcode{o.DCF('2M')} (lines 16 and 17)
because you don't know what the calc frequency is 
(it is determined outside).

\end{parchment}

