\section{Use of myfints\label{sec:myfints}}

This section covers the most important methods specific to
\myfints{}.
It does not enumerate all the methods of \myfints{} however.
For a complete list of methods, type either \mcode{methods(myfints\_obj)}
or \mcode{methods(myfints)}.

\subsection{Aligning Data}

\funentry{aligndata}
   Align inputted \mcode{myfints} objects both along time and fields. 
  
\usage
%
   \begin{lstlisting}[numbers=none]
   [ofts1, ofts2, ...] = aligndata(ifts1, ifts2, ...)
   [ofts1, ofts2, ...] = aligndata(ifts1, ifts2, ..., mode)
   [ofts1, ofts2, ...] = aligndata(ifts1, ifts2, ..., aligningdates)
   [ofts1, ofts2, ...] = aligndata(ifts1, ifts2, ..., mode, aligningdates)
   [ofts1, ofts2, ...] = aligndata(ifts1, ifts2, ..., aligningdates, mode)
   [ofts1, ofts2, ...] = aligndata(..., 'CalcMethod', calcMethod)
   \end{lstlisting}
%
\where
%
   \begin{argdesc}
	   \item[ifts1, ifts2,...] 
                      \myfints{} objects to be aligned. Multiple such objects
                      are allowed.
	   \item[ofts1, ofts2,...] 
                      returned \myfints{} objects that have been aligned.
                      Each corresponds to and has the same type of the inputted ones;
                      i.e., if the original one is of type \mcode{BKTOPRICE}, 
                      the corresponding aligned one is still of the same type.
	   \item[mode]  
                      control how the fields in time series aligned.
	                   Values can be one of
	                   \begin{itemize}
	                      \item \mcode{'union'}. include all (unique) fields occurred in all
	                                inputted \myfints{} objects. NaNs are filled
	                                for original nonexistent fields.
	                      \item \mcode{'intersect'}. only take the common fields in all
	                                inputted \myfints{} objects. Fields specific
	                                to a certain \myfints{} object are removed.
                      \end{itemize}
	                   \emph{DEFAULT} is \mcode{'intersect'}.
	   \item[aligningdates]
                      A vector of dates (numeric vector or cell vector of date
                      string) to be aligned,  OR\par
                      a char vector (string) or scalar number 
                      indicating aligned frequency
                      which can be
                      \begin{lstlisting}[numbers=none,backgroundcolor=\color{white}]
   1, DAILY, Daily, daily, D, d
   2, WEEKLY, Weekly, weekly, W, w
   3, MONTHLY, Monthly, monthly, M, m
   4, QUARTERLY, Quarterly, quarterly, Q, q							
   5, SEMIANNUAL, Semiannual, semiannual, S, s						
   6, ANNUAL, Annual, annual, A, a
\end{lstlisting}
\vspace*{-.3cm}
the same as \mcode{convertto(}) of \matlab{}'s fints.

                     
                     \begin{itemize*}
                     \item All the inputted \myfints{} objects will be aligned
                      against this date vector or specified frequency. 
                      Returned objects being aligned
                      have the \mcode{aligningdates} (when it is a date vector)
                      or equivalent (when it is a frequency) as their date field. 

                     \item For returned objects, on a certain date, 
                      data are the most recent available (known) data 
                      between the date (inclusive) and its previous date
                      (exclusive). 

                     If no data in inputted objects available 
                      in a period (specified or divided by \mcode{aligneddates}),
                      \mcode{NaN}s will be filled.

                     \item If NOT provided, aligned dates will be the common dates
                      in all inputted objects.
                     \end{itemize*}
	   \item[calcMethod]
	                  Following a char-type string indicator parameter \mcode{'CalcMethod'} (case-insensitive),
 specifies which method is used to calculate resampling sample data.
 Methods could be one of
 \begin{itemize}
 \item \mcode{'CumSum'}
 Returns the cumulative sum of the values between each sampling period (e.g., week, month, etc.)
 Data for missing dates are given the value 0.
 \item \mcode{'Exact'}
 Returns the exact value at the end-of-sampling-period date. No data manipulation occurs.
 \item \mcode{'Nearest'}
 (\emph{default}) Returns the values located at the end-of-sampling-period date. 
 If there is missing data, Nearest returns the nearest data point preceding the end-of-sampling-period date. 
 \item \mcode{'SimpAvg'}
 Returns an averaged value over each sampling period 
 that only takes into account dates with data (non NaN) within each quarter.
 \end{itemize}

   \end{argdesc}
%

\funentry{aligndates}

   aligns inputted \mcode{myfints} objects along time.

\usage
%
   \begin{lstlisting}[numbers=none]
   [ofts1, ofts2, ...] = aligndates(ifts1, ifts2, ...)
   [ofts1, ofts2, ...] = aligndates(ifts1, ifts2, ..., aligningdates)
   [ofts1, ofts2, ...] = aligndates(..., 'CalcMethod', calcMethod)
   \end{lstlisting}
%
   This function is similar to \mcode{aligndata}, except it aligns data only along date dimension.
   See the explanation of arguments there. 
\medskip

\funentry{alignfields}

   aligns inputted \mcode{myfints} objects along fields.
%
\usage
   \begin{lstlisting}[numbers=none]
   [ofts1, ofts2, ...] = alignfields(ifts1, ifts2, ...)
   [ofts1, ofts2, ...] = alignfields(ifts1, ifts2, ..., mode)
   \end{lstlisting}
%
   This function is similar to \mcode{aligndata}, except it aligns data only along fields dimension.
   See the explanation of arguments there. 
\medskip

\funentry{isaligneddata, isalignedfields, isaligneddates}
   Determine if \myfints{} objects given in arg list are aligned both/either in time and fields.
%
\usage
   \begin{lstlisting}[numbers=none]
   isaligned = isaligneddata(fts1, fts2, ...)
   isaligned = isalignedfields(fts1, fts2, ...)
   isaligned = isaligneddates(fts1, fts2, ...)
   \end{lstlisting}
\where
   \mcode{tfs1}, \mcode{tfs2}, etc. are \myfints{} objects to be determined. 
   If they are aligned both in  time and fields, returns true; otherwise false.

   Note that this function is different from \mcode{iscompatible(fts1,fts2)}
   which is inherited from \matlab's \mcode{fints} and returns true as long as the two 
   objects have the same fields and date irrespective of field order.
   
\subsection{Filling Missing Data}

\funentry{backfill}
   Fill time series backwards where the values are \mcode{NaN}s with previous latest available data.

\usage
   \begin{lstlisting}[numbers=none]
   fts = backfill(old_fts, lookbackperiod, mode)
   \end{lstlisting}

\inarg
   \begin{argdesc}
   \item [old\_fts]	  a \myfints{} object to be filled. 
   \item [lookbackperiod]   number of periods to look back for filling.
         If data at a time is \mcode{NaN}, 
         the function will look back 
         at most \mcode{lookbackperiod} periods
         and find the latest available data in these past periods to fill the \mcode{NaN}. 
         If you want to look back through to the very beginning of time series, 
         just put this value as \mcode{inf}.
   \item [mode] can be \mcode{'row'} or \mcode{'entry'}, 
         meaning filling occurs only when the whole row are \mcode{NaN}s or
         whenever an element is a \mcode{NaN}, respectively.
   \end{argdesc}

\outarg
   \begin{argdesc}
   \item [fts] the backfilled \myfints{} object.
   \end{argdesc}


\funentry{forwardfill}
   Fill time series forwards where the values are \mcode{NaN}s with most recent available data in 
   the future.

\usage
   \begin{lstlisting}[numbers=none]
    fts = forwardfill(old_fts, lookfwdperiod, mode)
   \end{lstlisting}

\inarg
   \begin{argdesc}
   \item [old\_fts]	  a \myfints{} object to be filled.
   \item [lookfwdperiod]   number of periods to look forward (into future) for filling.
         If data at a time point is \mcode{NaN}, 
         the function will look forward 
         at most \mcode{lookfwdperiod} periods
         and find the most recent available data in future these periods to fill the \mcode{NaN}. 
         If you want to look forward through to the last period, 
         just put this value as \mcode{inf}.
   \item [mode] can be \mcode{'row'} or \mcode{'entry'}, 
         meaning filling occurs only when the whole row are \mcode{NaN}s or
         whenever an element is a \mcode{NaN}, respectively.
   \end{argdesc}

\outarg
   \begin{argdesc}
	\item [fts] the forward filled \myfints{} object.
   \end{argdesc}


\subsection{Factor Backtesting}

\funentry{optimize}
 Calculates optimal weights for multiple alpha sources (factors) based on their historical performance
(in terms of IC).

In the following, we suppose there are $K$ factors and $N$ securities,
over a total of $T$ periods. 
For each factor, we may also include their lagged values in current period optimization.
(Current factor value always be included.)
We denote by $L_k$ the number of lagged factors involved for factor $k$,
where $k=1,\ldots,K$ indexing factor. 
So the total number of factors involved in optimization is $K+\sum_{k=1}^K L_k$.
By optimization, we are trying to obtain optimal weights for each factor (and their lagged ones)
over each period.
Denote factor weights at period $t$ by
   \begin{align}\label{eq:W}
     \Lbw'_t=(&w_{1,t},w_{1,t-1}\ldots,\allowbreak w_{1,t-L_1},\notag\\
             & \ldots \ldots,\notag\\
             & w_{k,t},w_{k,t-1}\ldots,\allowbreak w_{k,t-L_k},\\
             & \ldots \ldots,\notag\\
             & w_{K,t},w_{K,t-1}\ldots,\allowbreak w_{K,t-L_K}),\notag
   \end{align}
where $w_{k,t-\ell_k},k=1,\ldots,K,\ell_k=1,\ldots,L_k$ indicates optimal weights at period $t$ for $k$th factor at lag $\ell_k$.
Correspondingly, the factor vector at that period is denoted by
   \begin{align}\label{eq:F}
     \mathbf{F}_t=(&F_{1,t},F_{1,t-1}\ldots,\allowbreak F_{1,t-L_1},\notag\\
             & \ldots \ldots,\notag\\
             & F_{k,t},F_{k,t-1}\ldots,\allowbreak F_{k,t-L_k},\\
             & \ldots \ldots,\notag\\
             & F_{K,t},F_{K,t-1}\ldots,\allowbreak F_{K,t-L_K})'.\notag
   \end{align}
Both $\Lbw_t$ and $\mathbf{F}_t$ are $(K+\sum_{k=1}^K L_k)\times 1$ vectors.


\iffalse
The composited moving-average factor, denoted by $\mathbf{F}_t^{\text{c,ma}}$, is
   \begin{equation}\label{eq:Fcma}
       F_t^{\text{c,ma}} = \Lbw_t' \mathbf{F}_t.
	\end{equation}
For one security, it is a scalar. 
Collect all such values for all securities,
It is a $N\times 1$ vector of 
  \begin{equation}
   \mathbf{F}_{t}^{\text{c,ma}} = ({F}_{t,i}^{\text{c,ma}})_{i=1,\ldots,N}.
  \end{equation}
\fi

Finally, for late reference, we define
\begin{align*}
    \mathbf{C}_t &= \cov(\mathbf{F}_t),\\
    \mathbf{D}_t &= \cov(\mathbf{F}_t,\mathbf{F}_{t-1}) \equiv \E\left\{\left[\mathbf{F}_t-E(\mathbf{F}_t)\right]\left[\mathbf{F}_{t-1}-E(\mathbf{F}_{t-1})\right]\right\}.
\end{align*}

\usage
   \begin{lstlisting}[numbers=none]
   function [W, IR_optm, alpha, IC, exitflag] = ...
            optimize(factors, stockRet, 'option_name', option_value, ...)
   \end{lstlisting}

\inarg
  \begin{argdesc}
	\item[factors]  (a cell vector of \myfints{}) Cell vector of factors of size $K$,
                   each factor (a \myfints{}) is a $T\times N$ \myfints{} object.
                   All the \myfints{} objects in \mcode{factors} must be aligned.
   \item[stockRet] (\myfints{}) Securities returns, $T\times N$. 
                   Must be aligned to \myfints{} objects in \mcode{factors}.
                   \footnote{When doing forecasting for next-period, one-period forward returns make sense.}
	\item['option\_name', option\_value, ...] Pairs acting as named arguments providing additional information to the optimization process. 
	   \begin{longtable}[r]{>{\ttfamily}l<{} p{1.5cm} c p{9.5cm}}
   	      \textsf{\textbf{Options}} & \textsf{\textbf{Values}} & \textsf{\textbf{Default}} & \textsf{\textbf{Meanings}} \\
		      \toprule
		      \endfirsthead
   	      \textsf{\textbf{options}} & \textsf{\textbf{values}} & \textsf{\textbf{default}} & \textsf{\textbf{meanings}} \\
		      \toprule
		      \endhead
		      \bottomrule
		      \endfoot
		      \bottomrule
		      \endlastfoot
             \rowcolor{cone}
	          \mcode{method} & '\textit{Obj:Con}' & \mcode{'M/V:RHO'} &
                    A string consisting of 2 parts (objective and contraint) delimited by '\mcode{:}'. 
                    See explanation below. \\
             \rowcolor{ctwo}
		       \mcode{lag} & $\geq 0$, int & 11 & 
                    Indicates how many lagged factors involved in the optimization.
                    Should be a scalar or a vector of length equal to $K$ (number of factors).
                    If it is a scalar, the same lag will be applied to all factors;
                    if a vector, each element applies to the corresponding factor in \mcode{factors}.\\
             \rowcolor{cone}
		       \mcode{rho} & $[0,1]$ & 1.0 & 
                    Target value of the constaint specified in \mcode{method}.
                    Since all available constaints for now are some kind of correlation,
                    this parameter shoud be in range of 0 to 1.
                    See notes below for more details.\\
             \rowcolor{ctwo}
		       \mcode{lambda} & $\geq 0$ & 0 & 
                    This parameter appears in objective.\par
                    if \mcode{M/V}, it controls weight mass distribution among weights;\par
                    if \mcode{M-V}, it plays a risk-aversion parameter role. \\
             \rowcolor{cone}
		       \mcode{ICWin} & $\geq 0$, int & [36 0] & 
                    Specifies the period-window used for calculating mean and std of ICs, consisting of 2 values.\par
                    The first indicates how many past-period ICs used in optimization.
                    0 means \emph{expanding}, others the \emph{actual} window size;
                    should $>$ lag+2 (or 0) since we have \mcode{lag} lagged factors involved
                    and at the least 2 ICs needed to get a meaningful standard deviation.\par
                    The second specifies how many forward-period ICs included in calculating IC.
                    If want to include all future periods, set it to \mcode{Inf}.\\
             \rowcolor{ctwo}
             \mcode{lb} & $\geq 0$ & 0 &
                    Lower bound of all weights.
                    Note that the upper bound of all weights is 1.\\
             \rowcolor{cone}
             \mcode{IC} & {\tiny$T\times (K+\sum_{k=1}^K L_k)$}\par \myfints{} & empty &
                    IC matrix provided by user.
                    If ommitted or provided as empty (\mcode{[]}), 
                    \mcode{optimize} will calculate IC from 
                    \mcode{factors} and \mcode{stockRet}.
	   \end{longtable}
  \end{argdesc}

\outarg
  \begin{argdesc}
  \item[W]   (\myfints{}) optimal weights, $T\times (K+\sum_{k=1}^K L_k)$ where $T$ is number of periods, 
     $K$ the number of factors and $L_k$ (option \mcode{lag}) is the number of lags involved for factor $k$. 
     Each \emph{row} corresponds to a one-time period optimized weight vector, $\Lbw'_t$, given in equation (\ref{eq:W}).
  \item[IR\_optm] (\myfints{}) IR values when the \mcode{W} is applied to the objective function, $T\times 1$.
  \item[alpha] (\myfints{}) alphas when apply \mcode{W} to blend factors, $T\times N$.
  \item[IC] (\myfints{}) Information Coefficients for each factor (include lagged factors) at every time period, $T\times (K+\sum_{k=1}^K L_k)$.
            If user provided an IC via \mcode{'IC'} parameter, it will be the same as the user-provided; 
            otherwise, it will be the one \mcode{optimize} calculated from \mcode{factors} and \mcode{stockRet}.
  \item[exitflag] (\myfints{}) flags for diagnosing optimization process. 1 indicates success, 0 not convergent, negative values indicate
     problems. See also \texttt{fmincon}. $T\times 1$, each element corresponding to one period optimization.
  \end{argdesc}

\fundesc{Notes:}
\begin{enumerate*}
  \item \mcode{optimize} treats \mcode{NaN}s in \mcode{stockRet} and \mcode{factors} as missing values
       (excluded from optimization).
  \item All matrices and vectors returned have the same number of rows ($T$).
        The very first few rows (\mcode{max(lag)+2}) are \texttt{NaN}s (needed for IC calculation).
  \item Optimization method can be specified by named parameter \textbf{method}
        which consists of two parts delimited by a '\mcode{:}'. 
        The first part specifies the type of objective, the second for constraint.

        \begin{itemize*}
        \item \emph{Objective} could be one of the following:
	        \begin{enumerate*}
           \setlength{\itemsep}{6pt}
	        \item \mcode{'EW'}. Simply equally weight factors. No constraint should be specified.
	        \item \mcode{'M/V'}.
		       $\displaystyle
		       \max_{\Lbw_t}\,\IR_t = \frac{\Lbw_t'\,\overline{\mathbf{IC}}_t}
		                                     {\sqrt{\Lbw_t'\,\bm{\Sigma}_{\mathbf{IC_t}}\,\Lbw_t}}
		            - \lambda \Lbw_t'\Lbw_t.
		       $
             \footnote{Gradient (omitting subscript $t$):
             $\nabla\IR(\Lbw)=(\Lbw'\bm{\Sigma}_{\mathbf{IC}}\Lbw)^{-\frac{1}{2}}
             \left[\overline{\mathbf{IC}}_t-
             \Lbw'\overline{\mathbf{IC}}_t(\Lbw'\bm{\Sigma}_{\mathbf{IC}}\Lbw)^{-1}\bm{\Sigma}_{\mathbf{IC}}\Lbw\right].$}
	        \item \mcode{'M-V'}.
		       $\displaystyle
		       \max_{\Lbw_t}\,\IR_t = \Lbw_t'\,\overline{\mathbf{IC}}_t - 
		                                \lambda\left(\Lbw_t'\,\bm{\Sigma}_{\mathbf{IC}}\,\Lbw_t\right).
		       $
	        \end{enumerate*}

        \item \emph{Constraint} could be one of the following:
	        \begin{enumerate*}
           \setlength{\itemsep}{6pt}
	        \item \mcode{'RHO'}.
		       $\displaystyle
		        \frac{\cov\left(\Lbw_t'\mathbf{F}_t,\,\Lbw_{t}'\mathbf{F}_{t-1}\right)}
		         {\var\left(\Lbw_t'\mathbf{F}_t\right)} 
             = \frac{\Lbw_t' \mathbf{D}_t \Lbw_t}{\Lbw_t' \mathbf{C}_t \Lbw_t}\geq \rho.
		       $
             \footnote{Gradient (omitting subscript $t$):
             $\nabla\rho(\Lbw) =
             (\Lbw'\mathbf{C}\Lbw)^{-1}\left[ (\mathbf{D}+\mathbf{D}')-
             \frac{\Lbw'\,\mathbf{D}\,\Lbw}{\Lbw'\,\mathbf{C}\,\Lbw}(\mathbf{C}+\mathbf{C}') \right]\Lbw.$}

	      \item \mcode{'RA'}.
		       $\displaystyle
		        \frac{\cov\left(\Lbw_t'\mathbf{F}_t,\,\Lbw_{t-1}'\mathbf{F}_{t-1}\right)}
		         {\sqrt{\var\left(\Lbw_t'\mathbf{F}_t\right)\var\left(\Lbw_{t-1}'\mathbf{F}_{t-1}\right)}}
             =\frac{\Lbw_t'\mathbf{D}_t\Lbw_{t-1}}{\sqrt{\Lbw_t'\mathbf{C}_t\Lbw_t{\Lbw_{t-1}'\mathbf{C}_{t-1}\Lbw_{t-1}}}} \geq \rho.
		       $
             \footnote{Gradient (omitting subscript $t$):
             $\nabla\rho(\Lbw) =
              (\Lbw_{t-1}'\mathbf{C_{t-1}}\Lbw_{t-1}\;
               \Lbw'\mathbf{C}\Lbw)^{-\frac{1}{2}}\left[\mathbf{D}\Lbw_{t-1}-
               \Lbw'\,\mathbf{D}\,\Lbw_{t-1}(\Lbw'\,\mathbf{C}\,\Lbw)^{-1}\mathbf{C}\Lbw\right].$}

         \item \mcode{'RFAC'},
            $\displaystyle
               \Lbw'_t\,(\corr_t^{k,\ell_k})_{k=1,\ldots,K}^{\ell_k=0,\ldots,L_k} \geq \rho,
            $
          where $\corr_t^{k,l_k} = \corr(\mathbf{F}_t^{k,\ell_k},\,\mathbf{F}_{t-1}^{k,\ell_k})$ 
          and $\mathbf{F}_t^{k,\ell_k}$ is $k$th factor with lag $\ell_k$, $k=1,\ldots,K$, 
          $\ell_k=0,\ldots,L_k$. 
          $\ell_k=0$ means no lagging.
          
	      \item \mcode{'RHO\_EQ', 'RA_EQ', 'RFAC_EQ'} are corresponding equality constraints of above.
	      \end{enumerate*}
      \end{itemize*}
      Depending on different factors, different methods may have different performances. 
\end{enumerate*}

A final note on \mcode{optimize}.
\mcode{optimize} provide mathematical ways to optimally combine factors.
It is easy to substitute our intuition for the mathematics, but this can lead to trouble.
We should instead verify that the mathematics and our intuition agree: 
otherwise, either our intuition is wrong or our model is inadequate.
If our intuition and the mathematics of a model do not agree,
we should seek a reconciliation before proceeding.

\funentry{normalize}
  Normalize factors. Different methods allowed.

\usage
   \begin{lstlisting}[numbers=none]
   score = normalize(factor, 'option_name', option_value, ...)
   \end{lstlisting}

\inarg
   \begin{argdesc}
      \item[factor]  (\myfints{}) Factor to be normalized.
      \item['option\_name', option\_value, ...] (\myfints{}) Options providing additional information for normalization.
         \begin{argdesc}
            \item[method] Specifies which method used to do the normalization. 
                 Now two methods are allowed:
                 \begin{itemize}
                    \item \mcode{'norminv'}
                    \item \mcode{'zscore'}
                 \end{itemize}
                 \emph{Default} is \mcode{'zscore'}.
            \item[mode] Indicates if reverse the factor ranking orders. Two possible values:
                  \begin{itemize}
                    \item \mcode{'ascend'} The normalized result has the same ranking as the original.
                    \item \mcode{'descend'} The ranking of the normalized result is reversed.
                  \end{itemize}
                  \emph{Default} is \mcode{'ascend'}, which corresponds to 'the-higher-the-better' factors.
            \item[weight] (\myfints{} or ordinary matrix compatible to the \mcode{factor}) 
                  Indicates if corresponding elements in \mcode{factor} belong to the universe.
                  Weight value of 0 or \mcode{NaN} will be excluded the normalization process.
                  Other weight values treat the same.
                  \emph{Default} all element in \mcode{factor} will be involved.
            \item[GICS]  (\myfints{} or ordinary matrix compatible to the \mcode{factor})
                  If provided, normalization will also be \emph{neutralized}.
                  The value of \mcode{GICS} must be integers.
            \item[level] Controls to what level in \mcode{GICS} the neutralization is performed.

                 Standard GICS code consists of 8 digits, a form of like \mcode{25203010}.
                 The first 2 (\mcode{25} in the previous example) refer to \textbf{sector},
                 the next 2 (\mcode{20}) to \textbf{industry group},
                 the 5-6th (\mcode{30}) to \textbf{industry},
                 and the last 2 (\mcode{10}) to \textbf{sub-industry}.
                 Correspondingly, the value of level can be:
                 0: NO neutralization;
                 1: sector;
                 2: industry group; 
                 3: industry;
                 or 4: sub-industry.
                 \emph{Default} is 1.
                 Note that it is only effective when \mcode{GICS} being provided.

                 \mcode{level} can also be \mcode{'customized'}, 
                 used with non-standard \mcode{GICS} passed by user.
                 In this case, the neutralization will be performed based on all digits of 
                 the non-standard gics code.

         \end{argdesc}
   \end{argdesc}

\funentry{neutralize}
  Neutralizes specified function.

\usage
   \begin{lstlisting}[numbers=none]
   res = neutralize(factor, GICS, fun)
   res = neutralize(factor, GICS, fun, level)
   \end{lstlisting}

\inarg
   \begin{argdesc}
      \item[factor] (\myfints{}) Factor on which neutralization is performed.
      \item[GICS]   (\myfints{} or ordinary matrix compatible to the \mcode{factor})
                    Neutralization is based on this parameter. 
                    If not provided, \mcode{neutralize} simply perform \mcode{fun} on \mcode{factor}.
                    The value of \mcode{GICS} must be integers.

      \item[fun]    (function handle) Specifies operation to be neutralized.
                    It should be an ordinary \matlab{} function of form:
   \begin{lstlisting}[numbers=none]
         function out_matrix = fun(input_matrix)
   \end{lstlisting}
                    where \mcode{input_matrix} and \mcode{out_matrix} are ordinary matrices with the
                    same dimension. 
                    \emph{Whatever it to do, \mcode{fun} should ignore \mcode{NaN}s.}
       \item[level] Controls to what level in \mcode{GICS} the neutralization is performed.

                 Standard GICS code consists of 8 digits, a form of like \mcode{25203010}.
                 The first 2 (\mcode{25} in the previous example) refer to \textbf{sector},
                 the next 2 (\mcode{20}) to \textbf{industry group},
                 the 5-6th (\mcode{30}) to \textbf{industry},
                 and the last 2 (\mcode{10}) to \textbf{sub-industry}.
                 Correspondingly, the value of level can be:
                 0: NO neutralization;
                 1: sector;
                 2: industry group; 
                 3: industry;
                 or 4: sub-industry.
                 \emph{Default} is 1.
                 Note that it is only effective when \mcode{GICS} being provided.

                 \mcode{level} can also be \mcode{'customized'}, 
                 used with non-standard \mcode{GICS} passed by user.
                 In this case, the neutralization will be performed based on all digits of 
                 the non-standard gics code.

   \end{argdesc}

\outarg
   \begin{argdesc}
      \item [res]  (\myfints{}) Result returned.
   \end{argdesc}

\funentry{csregress}
   Cross-sectional regression of a factor (\mcode{yfactor}, regressor) against a group of other factors
  (\mcode{xfactors}, regressands).

\usage
   \begin{lstlisting}[numbers=none]
   [beta, epsilon, std_beta, R2, mse] = csregress(yfactor, xfactors, 'option_name', option_value, ...)
   \end{lstlisting}

\inarg
   \begin{argdesc}
   \item[yfactor]   (\myfints{}) $T\times N$, factor used as regressor (dependent variable). 
                    $T$ is the number of periods (or time points), 
                    $N$ the number of observations at each period.
   \item[xfactors]  (a cell vector of \myfints{}) Contains $K$ \myfints{},
                    each is $T\times N$, 
                    factors used as regressands (independent variables)
   \item['option\_name', option\_value, ...] named arguments providing additional information.
        \begin{argdesc}
        \item[weight] (\myfints{} or ordinary matrix compatible to the \mcode{yfactor}) 
                      $T\times N$. 
                      If provided, Weighted Least square will be used to solve the regression problem;
                      otherwise, ordinary least square (weight matrix is identity).
        \end{argdesc}
   \end{argdesc}

  \mcode{yfactor} and elements of \mcode{xfactors} must be aligned. 
  If \mcode{weight} provided, it also must be aligned to \mcode{ycode} 
  (in case of \mcode{weight} is a \myfints{}) or compatible to it 
  (in case of \mcode{weight} is a matrix).
  The regression is performed at each time period as
  \[
      w_i^{\frac{1}{2}}\,y_i = w_i^{\frac{1}{2}}\,\sum_{j=1}^{K} \beta_{j}x_{ij} + \epsilon_i, \quad i=1,\ldots,N
  \]
  where $N$ is the number of cross-sectional observations (i.e., number of fields in \mcode{yfactor})
  and each of \mcode{xfactors}, $K$ is the number of elements in \mcode{xfactors},
  $y_i, i=1,\ldots,N$ is a row from \mcode{yfactor}, 
  $w_i, i=1,\ldots,N$ is a row from \mcode{weight}, 
  $x_{i,j}, i=1,\ldots,N$ is the corresponding row from \mcode{xfactors\{j\}}.

\iffalse
  In matrix form, the regression is (at a period)
  \[
     \bm{W}^{\frac{1}{2}}\bm{y} = \bm{W}^{\frac{1}{2}}\bm{X}\bm{\beta} + \bm{\epsilon},
  \]
  where
  \begin{equation*}
     \bm{W}^{\frac{1}{2}} = \begin{pmatrix}
                             \sqrt{w_1} & &\\
                              &\ddots &\\
                              & &\sqrt{w_N} \\
                            \end{pmatrix},\;
     \bm{y} = \begin{pmatrix} y_1\\
                              \vdots\\
                              y_N\\
              \end{pmatrix},\;
     \bm{X} = \begin{pmatrix}
               x_{11} & \cdots & x_{1K}\\
               \vdots & \ddots & \vdots\\
               x_{N1} & \cdots & x_{NK}\\
              \end{pmatrix},\;
     \bm{\beta} = \begin{pmatrix}
                  \beta_1\\
                  \vdots\\
                  \beta_K
                  \end{pmatrix}
     \bm{\epsilon} = \begin{pmatrix}
                     \epsilon_1\\
                     \vdots\\
                     \epsilon_N
                    \end{pmatrix},
  \end{equation*}
\fi  

  If you want to include a constant term in regression, you should provide it as a \myfints{} element
  in \mcode{xfactors}. A simple way to do this is
  \begin{lstlisting}[numbers=none]
   constant_x = yfactor;
   constant_x(:,:) = 1;
  \end{lstlisting}
  Then you can add this constant \myfints{} into \mcode{xfactors}.

\outarg
   \begin{argdesc}
   \item[beta]     (\myfints{}) $T\times K$, corresponding to the estimated coefficients of $\beta$.
   \item[epsilon]  (\myfints{}) $T\times N$, regression residual.
   \item[std\_beta](\myfints{}) $T\times K$, stdandard error of \mcode{beta}.
   \item[R2]       (\myfints{}) $T\times 1$, $R^2$, goodness of fit measure.
   \item[mse]      (\myfints{}) $T\times 1$, Mean-Squared-Error of regression,
                   element at time $t$ corresponding to the regression at that time. 
   \end{argdesc}

A common use of \mcode{csregress} is regressing stock returns on a group of factors
to get factor returns. In this case, stock return is \mcode{yfactor},
every factor is a element (\myfints{} object) of \mcode{xfactors}.

\funentry{csqtrtn}
    Calculate quantile returns based on given signal.

\usage
   \begin{lstlisting}[numbers=none]
   [qtrtn, qtweight] = csqtrtn(signalfts, returnfts, 'option_name', option_value,...)
   \end{lstlisting}

\inarg
   \begin{argdesc}
	   \item[signalfts] (\myfints{} object) time series of signal values
	   \item[reutrnfts] (\myfints{} object) time series on which quantile returns are going to be calculated
	   \item['option\_name', option\_value,...] optional arguments.
                     Available options are:
		\begin{argdesc}
			  \item[weight] a \myfints{} object or an ordinary matrix compatible to \mcode{returnfts},
			               indicates the weights for corresponding elements in \mcode{returnfts},
			              \emph{default} equal weight
		    \item[univ] a \myfints{} object or an ordinary matrix compatible to \mcode{returnfts}, 
                     indicating
		               whether the stock is in the universe at each time point.
		               if specified, the quintile returns will be
		               only calculated for stocks within the universe at each point in time. 
			  \item[GICS] a \myfints{} object or an ordinary matrix compatible to \mcode{returnfts},
			             contains GICS code corresponding to elements in \mcode{returnfts}.
                      Value of \mcode{GICS} must be integer.
                      If provided, the quantile spreads are calculated within each sector identified by \mcode{GICS};
			             if \emph{omitted}, the quantile return are calculated on the whole universe.
		    \item [qtile] Specifies the quantile points. \emph{Default} is quintile (i.e., [0, 0.2, 0.4, 0.6, 0.8, 1])
		    \item [level] Controls to what level in \mcode{GICS} the neutralization is performed.

                 Standard GICS code consists of 8 digits, a form of like \mcode{25203010}.
                 The first 2 (\mcode{25} in the previous example) refer to \textbf{sector},
                 the next 2 (\mcode{20}) to \textbf{industry group},
                 the 5-6th (\mcode{30}) to \textbf{industry},
                 and the last 2 (\mcode{10}) to \textbf{sub-industry}.
                 Correspondingly, the value of level can be:
                 0: NO neutralization;
                 1: sector;
                 2: industry group; 
                 3: industry;
                 or 4: sub-industry.
                 \emph{Default} is 1.
                 Note that it is only effective when \mcode{GICS} being provided.

                 \mcode{level} can also be \mcode{'customized'}, 
                 used with non-standard \mcode{GICS} passed by user.
                 In this case, the neutralization will be performed based on all digits of 
                 the non-standard gics code.

	   \end{argdesc}	

	   The quantile returns are calculated as weighted average of part of
	   \mcode{returnfts} between certain quantiles.
	\end{argdesc}

\outarg
   \begin{argdesc}	
	   \item[qtrtn] (\myfints{} object)  time series of quantile returns
	   \item[qtweight] (cell vector of \myfints{} objects) time series of quantile portfolio weights
   \end{argdesc}


\funentry{csqtspread}
    Calculate quantile spreads based on given signal.

\usage
   \begin{lstlisting}[numbers=none]
   [ofts, weight] = csqtspread(signalfts, returnfts, 'option_name', option_value,...)
   \end{lstlisting}

\inarg
   \begin{argdesc}
	   \item[signalfts] (\myfints{} object) time series of signal values
	   \item[reutrnfts] (\myfints{} object) time series on which quantile
	                  spreads are going to be calculated
	   \item['option\_name', option\_value,...] optional arguments controlling
	                  how the portfolio be constructed.
                     Available options are:
		\begin{argdesc}
			  \item[window] an integer, indicates the window size when calculating
			               moving sums, \emph{default} \mcode{inf}
			  \item[weight] a \myfints{} object or an ordinary matrix compatible to \mcode{returnfts},
			               indicates the weights for corresponding elements in \mcode{returnfts},
			              \emph{default} equal weight
		     \item[univ] a \myfints{} object or an ordinary matrix compatible to \mcode{returnfts}, 
                     indicating
		               whether the stock is in the universe at each time point.
		               If specified, the quintile spreads will be
		               only calculated for stocks within the universe at each point in time. 
			  \item[GICS] a \myfints{} object or an ordinary matrix compatible to \mcode{returnfts},
			             contains GICS code corresponding to elements in \mcode{returnfts}.
                      Value of \mcode{GICS} must be integer.
                      If provided, the quantile spreads are calculated within each sector identified by GICS;
			             if \emph{omitted}, the quantile spreads are calculated on the whole universe.
           \item[level] Controls to what level in \mcode{GICS} the neutralization is performed.

                 Standard GICS code consists of 8 digits, a form of like \mcode{25203010}.
                 The first 2 (\mcode{25} in the previous example) refer to \textbf{sector},
                 the next 2 (\mcode{20}) to \textbf{industry group},
                 the 5-6th (\mcode{30}) to \textbf{industry},
                 and the last 2 (\mcode{10}) to \textbf{sub-industry}.
                 Correspondingly, the value of level can be:
                 0: NO neutralization;
                 1: sector;
                 2: industry group; 
                 3: industry;
                 or 4: sub-industry.
                 \emph{Default} is 1.
                 Note that it is only effective when \mcode{GICS} being provided.

                 \mcode{level} can also be \mcode{'customized'}, 
                 used with non-standard \mcode{GICS} passed by user.
                 In this case, the neutralization will be performed based on all digits of 
                 the non-standard gics code.

		     \item[long] a number between [0,1], indicating long stocks
                     with returns greater than 80\% of quantile returns among all stocks. 
                     Default is 0.8.
		     \item[short] a number between [0,1], indicating short stocks
                     with returns less than 20\% quantile returns among all stocks. 
                     Default is 0.2
	   \end{argdesc}	
%
	   The quantile spreads are calculated as weighted average of part of
	   \mcode{returnfts} greater than \mcode{long} quantile, minus weighted average of
	   part of \mcode{returnfts} less than \mcode{short} quantile. 
     
      Clearly, \emph{by default}, it's quintile spread (Q5$-$Q1).
      For \emph{long-only} portfolio, just set 'long' 0 and short 1. 
	\end{argdesc}

\outarg
   \begin{argdesc}	
	   \item[ofts] (\myfints{} object)  time series of quantile spreads and its moving sum
	   \item[weight] (\myfints{} object) time series of weight associated with \mcode{returnfts}
   \end{argdesc}


\funentry{tsplot}
Visulizes a \myfints{} object.

\usage
   \begin{lstlisting}[numbers=none]
   h = tsplot(fts, 'parameter',value,...)
   \end{lstlisting}

\inarg
\begin{argdesc}
\item[fts] (\myfints{}) Object to be plotted. 
\item ['parameter',value,$\ldots$] pairs acting as named arguments providing additional information for plotting the time series. 
      \begin{argdesc}
      \item[group] a matrix, where 
         \begin{itemize}
           \item number of columns is equal to the number of time series in \mcode{fts}
                 (i.e., number of fields (columns) of \mcode{fts} when it's a matrix);
           \item number of rows indicates how many subplots be drawn in the plot;
           \item its elements can be 0 (not draw the corresponding series in the subplot),
                 1 (draw the corresponding series in the corresponding subplot with y-axis on the left),
				 or -1 (draw the corresponding series in the corresponding subplot with y-axis on the right).
         \end{itemize}
       For example, for a 3-column \mcode{fts} which contains 3 time series,
       \mcode{group} can be 
  \begin{lstlisting}[numbers=none]
  [1 0 0; 0 1 0; 0 0 1]
  \end{lstlisting}
       which means the plot will have 3 subplots and every plot for a series. 
       Also \mcode{group} can be \mcode{[1 1 1]} which means drawing all 3 series in one subplot. \par
       Other examples of valid values for \mcode{group}:
  \begin{lstlisting}[numbers=none]
  [1 0 1; 0 1 1]                % 2 sub plots where the 3rd series used in both 
  [1 0 0; 1 0 0; 1 0 0; 1 0 0]  % The same 4 subplots where only 1st series used

  \end{lstlisting}
       \emph{Default} value 
       is row vector with all elements being one and length equal to number of column of \mcode{fts}; 
       i.e., all series will be drawn in one plot.
      \item[title]      A string or a cell string vector used as the caption of plots.
                        \begin{itemize}
                         \item If it is a string, all subplots will use it as caption;
                         \item if it is a vector, the length of it should be equal to the number of groups (implicitly specified by \mcode{group})
                               or 1 (all subplots share the same title as caption.)
                         \item If not provided, no caption will be shown in the plot or subplots.
                        \end{itemize} 
      \item[ylabel]     A string or a cell string vector used as the label for y-axis of plots. 
	                    The number of rows of the cell should be 1 (will be expanded to the number of subplots) or equal to number of subplots implied by \mcode{group};
						the number of columns of the cell can be 1 (only for left y-axis) or 2 (for left- and right-y axes).
						Right-y axes only exist when -1 exists in \mcode{group}; otherwise, the second column is useless.
      \item[ycolor]     A cell of 3-element (representing RGB color) vectors like 
  \begin{lstlisting}[numbers=none]	  
	  {[1 0 0] [0 1 0]; [1 0.4 0] [0 0 1]}
  \end{lstlisting}
	                    used as the color for y-axis of plots. 
	                    The number of rows of the cell should be 1 (will be expanded to the number of subplots) or equal to number of subplots implied by \mcode{group};
						the number of columns of the cell can be 1 (only for left y-axis) or 2 (for left- and right-y axes).
						Right-y axes only exist when -1 exists in \mcode{group}; otherwise, the second column is useless.
      \item[notes]      A string or a cell string vector that will be shown just below every subplot as notes. 
                        Its format is similar with \mcode{title}.
      \item[dateformat] A string specifying date format shown along the x-axis. 
                        Default is \mcode{'yy-mm'}.
      \item[drawfun]    A cell vector of function handles specifying how data be plotted 
                        (such as \mcode{@bar}, \mcode{@plot}, etc.) 
                        It can be any length since it will be used rotatively.
                        \emph{Default} is \mcode{\{@plot\}}, i.e., every series will be drawn as a curve.
      \item[style]      A cell string vector specifying line style to be drawn in plots. 
                        Its can be of any length since it will be used rotatively,
                        but need to be properly corresponding to \mcode{drawfun};
                        every element of the vector can be any form the corresponding \mcode{drawfun} allowed.\par
                        \emph{Default value} is \mcode{\{'-b', '-g', '-r', '-k', '-c', '-m', '-y'\}},
                        meaning solid line with different colors and line width is \matlab's default value.
                        This is appropriate since default \mcode{drawfun} is \mcode{@plot}.\par
                        More sophisticated example is \par
                        \mcode{\{\{'-b', 'linewidth', 2\}, \{'-.', 'Color', [1 0.4 0], 'Marker', '+'\}\}}. \par
                        Note that \mcode{[1 0.4 0]} is the ING \textcolor{ingorange}{\textbf{orange}}.
                        See \matlab{} help on specific draw functions about valid properties. 
      \item[ymax]       a scalar or vector specifying maximum value of y-axis. 
                        If it is a scalar, every subplot will have the same \mcode{ymax}; 
                        otherwise its length must be the same as number of subplots implicitly specified by \mcode{group},
                        in this case every element corresponds to a subplot.\par
                        If not provided (\emph{default}) or is \mcode{NaN}, 
                        the function will automatically use maximum value of series used in a subplot 
                        as \mcode{ymax} for that subplot.
      \item[ymin]       Similar as \mcode{ymax}, except it is minimum value of y-axis. 
      \item[hornlineposn] a vector specifying horizontal lines shown in \emph{every} subplot. 
                        So the value of its elements are treated as y-coordinates. 
                        The length of the vector is the number of lines to be drawn.
                        These line will shown in red color with width of 1.
      \item[vertlineposn] Similar with \mcode{hornlineposn} except it is for vertical line. 
                        The value of its elements are treated as x-coordinates. 
      \item[legend]    A cell used to control how legends be plotted. 
                         \begin{itemize}
                         \item If it only has one element (i.e., one string), all the subplots will share the same legends.
                         \item Otherwise, its length should be equal to the number of subplots implicitly specified by \mcode{group}.
                         \end{itemize}
                         Examples:\par
                         \mcode{\{\{'Universe', 'Sector'\}\}}, will be used in every subplot,
                         or \mcode{\{'Universe', 'Sector'\}}, 
                         where \mcode{'Universe'} will be used in first subplot, 
                         and \mcode{'Sector'} in second subplot, 
                         suppose only 2 suplots be specified by \mcode{group} (otherwise an error occurs).
                         
     \item[layout]    A $1\times2$ vector specifying the subplot layout. For example,
                      \mcode{[3,2]} means the there are total of 6 ($3\times 2$) sub-plots,
                      arranged in 3 rows - 2 columns manner 
                      and counted from left to right, then top to bottom.
                      The number of subplot determined by \mcode{layout} must be great than or equal to
                      the number of groups implied by \mcode{group}.
                      Default is \mcode{[<number of group>,1]}, meaning number of subplots equal to number of groups,
                      and every subplot holds a whole line.
     \item[range]     A cell vector of length equal to number of subplots (implicitly specified by \mcode{group}),
                      with each element is a vector or matrix, specifying the position and area the subplot corresponding to group
                      (corresponding to) will hold. 
                      Default is \mcode{1,2,3,...<number of group>}, meaning the $i$th group occupies the $i$th subplot.

                      For better understanding of \mcode{layout} and \mcode{range}, refer \matlab{}'s help on \mcode{subplot}.
	 \item[figure]    Specifies a figure handle on which the plots will be ploted.
	                  If not specified (which is default), a new figure will be created and used.
					  Specifying an exist figure handle, so we can draw on the same figure with different \mcode{tsplot} calls
					  (by specifying the same \mcode{layout} and different \mcode{range}).
	 \item[orientation] Can be  \mcode{portrait} or \mcode{landscape}. Default is \mcode{portrait}.
    \end{argdesc}
\end{argdesc}

\outarg
\hspace{0.5cm} Return the figure handle it plotted.

\fundesc{Examples:}

In its simplest form, this function can be called as
\begin{lstlisting}[numbers=none]
   tsplot(fts)
\end{lstlisting}
Every field in \mcode{fts} will be treated as a series and drawn together in one plot using default line style, 
without caption and y-axis label, and x-axis shown with dates in format of \mcode{'yy-mm'}.

\begin{lstlisting}[numbers=none]
   tsplot(fts, 'group', [1 0 0; 0 1 0, 0 0 1])
\end{lstlisting}
Suppose \mcode{fts} has 3 field (3 series), 
then this call will create a figure with 3 subplots, every subplot draws a series with default specifications.

\begin{lstlisting}[numbers=none]
   tsplot(fts, 'group', [1 0 0; 0 1 1], 'style', {'-g', {'.r', 'linewidth', 1}})
\end{lstlisting}
Suppose \mcode{fts} has 3 field (3 series),
then this call will create a figure with 2 subplots: the first draws the first column in \mcode{fts},
the second draws two line corresponding to the second and third column in \mcode{fts}, 
and in second subplot, the line corresponding to the third column of \mcode{fts} will be draw in dotted red line of width 1.

The \emph{font shape}, \emph{font size} and \emph{text color}
(and more, such as Greek letters, formulas typesetting) in any text-type options 
(title, labels, notes, legends, etc.)
of plots (or subplots) can also be changed 
by mixing \TeX{} directives in the string.
Suppose you want add some notes in a figure, set note string as
\begin{lstlisting}[numbers=none]
   note = '\fontsize{12}\bf\color[rgb]{1 .4 0}This is a note example.';
\end{lstlisting}
This set the note contents \mcode{'This is a note example.'}, 
{\large{}font size 12}, font shape \textbf{bold face}, font color \textcolor{ingorange}{orange}.
For more \TeX{} directives, see \matlab{} help on \texttt{text properties}.
Then you can call plot as normal:
\begin{lstlisting}[numbers=none]
   tsplot(fts, 'group', [1 0 0; 0 1 0, 0 0 1], 'notes', note);
\end{lstlisting}
In this example, there will be 3 subplots;
given only one note string, the same note
will be displayed at the bottom of each subplot.
If you want to different notes for different subplots,
just pass 3 notes as a cell vector such as
\begin{lstlisting}[numbers=none]
   tsplot(fts, 'group', [1 0 0; 0 1 0, 0 0 1], 'notes', {'note1', 'note2', 'note3'});
\end{lstlisting}


\funentry{mat2fts}
   Convert data from \matlab's vectors into \myfints{} objects.

\usage
   \begin{lstlisting}[numbers=none]
   fts = mat2fts(date, val, sid)
   fts = mat2fts(date, val, sid, fid)
   \end{lstlisting}

\inarg
   \begin{argdesc}
   \item[date, val] vectors represent dates and corresponding values, receptively.
   \item[sid]  vector represents individual ids (e.g., securities ids).
   \item[fid]  (optional) vector represents group ids (e.g., factor ids).
               If omitted, all data in \mcode{val} belong to the same group.
   \end{argdesc}

   Note that the sizes of \mcode{date}, \mcode{val}, \mcode{sid} and \mcode{fid}
   must be agree with each other.
   The combination of \mcode{date}, \mcode{sid}, \mcode{fid} must be unique.
   The orders of these vectors are irrelevant (ordering will be done inside \mcode{mat2fts}).
   
\outarg
   \begin{argdesc}
   \item[fts] a cell vector of \myfints{} object corresponding to each \mcode{fid}.
              The number of \myfints{} in \mcode{fts} is determined by number of unique values in \mcode{fid},
              the number of fields in each \myfints{} is determined by number of unique values in \mcode{sid},
              and number of periods (time points) in each \myfints{} is determined by number of 
              unique values in \mcode{date}.
              You can access the \mcode{fid} for each returned \myfints{} in \mcode{fts} by
                 \mcode{fts\{i\}.desc}.
   \end{argdesc}

 
\funentry{xls2fts}
   Convert data from Microsoft Excel sheet into \myfints{} objects.

\usage
   \begin{lstlisting}[numbers=none]
   [fts,...] = xls2fts(fileName, sheetName, dateColRange, sidColRange, valColsRange)
   \end{lstlisting}

\inarg
   \begin{argdesc}
   \item[fileName]  Name of an Excel document file from which data is read.
   \item[sheetName] Sheet name inside the \mcode{fileName} excel file.
   \item[dateColRange] a string, specify a range where data will be used as date. 
         Range should be in vecor form; i.e., either one row or one column, like \mcode{B2:B10000}.
   \item[sidColRange] a string, range where data be used as security id.
         Range should be in vecor form; i.e., either one row or one column, like \mcode{C2:C10000}.
   \item[valColsRange] a string or cell vector of string, specifies data used as values.
        Every string corresponds to a final \myfints{} returned.
   \end{argdesc}

   Note all the range spcified should be agree with each other in dimension.
   The combination of content from \mcode{dateColRange} and \mcode{sidColRange} must be unique.
   The occurance order is irrelevant.
   
\outarg
   \begin{argdesc}
   \item[fts, ...] one or more \myfints{} objects corresponding to each element in \mcode{valColsRanges}.
              The number of fields in each \myfints{} is determined by number of unique values in \mcode{sidColRange},
              and number of periods (time points) in each \myfints{} is determined by number of 
              unique values in \mcode{dateColRange}.
              You can get the corresponding \mcode{valColsRange} string for each returned \myfints{} in \mcode{fts} by
                 \mcode{fts\{i\}.desc}.
   \end{argdesc}
 

\subsection{Comparison Operators: \texttt{>=}, \texttt{>}, \texttt{==}, \texttt{$\sim$=}, \texttt{<=}, \texttt{<}}

      These operators perform element-by-element comparison between two \mcode{myfints} objects
      or one \mcode{myfints} object and an ordinary \matlab{} matrix (or scalar).
      You can compare two \mcode{myfints} objects like:
      \footnote{Note that all the factors are derived from \mcode{myfints}, 
      so these operators also apply to factors.
      For a complete methods of a class, type \mcode{methods(obj)} or \mcode{methods(class name)}
      from \matlab{} command window.}
      \begin{lstlisting}[numbers=none]
         result= obja > objb;
      \end{lstlisting}
      The \mcode{result} is of same type as of \mcode{obja}.
      You cOan also compare a \mcode{myfints} object with an ordinary matrix or scalar, 
      as long as they are compatible in term of matrix dimension, like this
      \begin{lstlisting}[numbers=none]
         result = 5 > obja;
      \end{lstlisting}
         or
      \begin{lstlisting}[numbers=none]
         result = obja < 5;
      \end{lstlisting}
      In this case, \mcode{result} is of the same type as \mcode{obja}.
      The general rule for type of returned result is the \mcode{result} 
      has the same type as of the most left of \mcode{myfints} objects.

      For compare if two myfints object equal as a whole (return a single true or false),
      you can use \mcode{isequal(ftsa, ftsb)} function, just as for normal \matlab{} matrix.

\subsection{Ordinary Matrix Algebra Operators: \texttt{+, -, *, /, .*, ./}}

     For \myfints, \mcode{*} and \mcode{/} are identical to \mcode{.*} and \mcode{./}, respectively.
     All other operators have the same meanings as those for \matlab{} matrices. 
	 All these operators can be used with mixing a \myfints{} object and ordinary matrix in any order.

\subsection{Indexing}

Not only does \mcode{myfints} allow the same indexing methods supported by \matlab's \mcode{fints},
it also support \matlab's matrix indexing methods.
The following examples summarize various indexing forms possible.

\begin{itemize*}
\item index rows (values of all fieldnames on the same date):
  \begin{lstlisting}[numbers=none]
     a = fts('01-Apr-2010'); 							
     a = fts('01-Apr-2010',:);  % equivalent to the line above  
     a = fts('01-Apr-2010::01-Apr-2011'); 		
     a = fts({'01-Apr-2010', '01-May-2010', '01-Oct-2010'},:); % must be with ,:
	 % Below union of dates specified used as index (sequence irrelevent)
	 a = fts({'2011-01-01::2011-05-31', '2011-03-21', '2009-12:21-2011-01-02'}, :); 
     a = fts({734100 734150 734200}, :); % numeric in cell used as dates
	 % Below combination of string & numeric type indexing
     a = fts({'2011-01-01::2011-05-31', num2cell(now-5:now), '2011-03-21'}, :); 
     a = fts(5); 		 % numeric in ordinary vector used as positioning index										
     a = fts(5,:);       % equivalent to the line above
     a = fts(5:7);         							
     a = fts(5:7,:);     % equivalent to the line above
     a = fts([5 7]);                      		
     a = fts([5 7],:);   % equivalent to the line above
	 a = fts(logical([1 0 0 1]),:);  % equivalent to fts([1 4],:); logical vector used as TF index
 \end{lstlisting}
\item index columns (values of fields on all dates):
  \begin{lstlisting}[numbers=none]
     a = fts.fieldname;
     a = fts(:,'fieldname');
     a = fts(:,{'fieldname1', 'fieldname2'});
     a = fts(:,2);
     a = fts(:,[2:10]);
  \end{lstlisting}
\item index columns and rows
  \begin{lstlisting}[numbers=none]
     a = fts.fieldname('01-Apr-2010'); 
     a = fts.fieldname(3:5);
     a = fts.fieldname(4);
     a = fts('01-Apr-2010','fieldname3');
     a = fts(3, 'fieldname2');
     a = fts({'01-Apr-2010', '01-Apr-2011'}, {'IBM', 'C', 'APPL'});
     a = fts({'01-Apr-2010', '01-Apr-2011'}, 1:30);
     a = fts('01-Apr-2010::01-Apr-2011', 1:30);
     a = fts(2:5, 3:8);
     a = fts(fts > 0.5);         % return a vector
     a = fts(fts == anotherfts); % must be aligned 
     a = fts(matrix);            % must compatible in size and contents
     a = fts(anotherfts);        % must be aligned and compatible in size and contents
 \end{lstlisting}
\item index the whole thing
  \begin{lstlisting}[numbers=none]
     fts(:,:);
     fts('::');
  \end{lstlisting}
\item modify \myfints{} object by indexing
  \begin{lstlisting}[numbers=none]
     % Remove some dates
     fts(3:5,:) = [];            
     fts({'01-Apr-2010', '01-Apr-2011'},:) = []; 
     % Remove some fields
     fts.IBM = [];
     fts(:, 'IBM') = [];
     fts(:, {'MSFT', 'GOOD'}) = [];
	 fts(:, '0059AAPL') = [];
	 fts.('0059AAPL') = [];    % NOTICE this usage when field is not valid matlab identifier
     fts(:, 3:5) = [];
  \end{lstlisting} 
  Removing this way must be about entire rows or columns, otherwise \matlab{} will complain.
  
  Assigning to irregular part of a \myfints{} object with compatible data is possible, like
  \begin{lstlisting}[numbers=none]
     fts(fts > 0.5) = 0;
     fts(fts > anotherfts) = -fts(fts > anotherfts); % flip signs
  \end{lstlisting}
\end{itemize*}
   
  The returned from indexing a \myfints{} object usually still is a \myfints{} object
  with appropriate dates and fieldnames labeled. 
  However, for indexing like \mcode{fts>0.5} the returned is ordinary \matlab{} vector
  since it is irregular. 
  In fact, this kinds of index mostly used in this way: \mcode{fts(fts>0.5)=100;}

  \mcode{end} can be appeared whenever it is appropriate; 
  i.e., when idexing with numeric values, like
  \begin{lstlisting}[numbers=none]
      a = obja(3:end);
  \end{lstlisting}
  return the 3rd to last rows of \mcode{obja} and return a object (\mcode{a}) of the 
  same type as of \mcode{obja}.

  \emph{The principle} is: when the first dimension is a cell vector or char-string, it is treated as dates, 
  no matter what's inside (i.e., date strings or numeric dates). 
  If the first dimension index is a numeric/logical vector, it is treated as position/true-false index.

  \emph{A key difference} between first (date)-dimension and second (field)-dimension index: 
  no matter what order you specified for dates, dates will always be ordered ascendingly; 
  while filed sequence will always be the same as you specified in index.

\subsection{Cross-Sectional Statistics}

\funentry{cscorr, csrankcorr, cscorrwt}
    Calculate the cross sectional correlation time series of two aligned \myfints{}.

\usage
   \begin{lstlisting}[numbers=none]
   ofts = cscorr(iftsA, iftsB, 'argname', value,...)
   ofts = csrankcorr(iftsA, iftsB)
   ofts = cscorrwt(iftsA, iftsB, iftsWt)   
   \end{lstlisting}

\inarg
   \begin{argdesc}
	\item[iftsA]	(\myfints{} object) One inputs those cross sectional correlation is to be found
	\item[iftsB]	(\myfints{} object) One inputs those cross sectional correlation is to be found
   \item['argname', value,...] specifies one or more optional name/value pairs. 
        Specify name inside single quotes. 
        The following table lists valid parameters and their values.
		   \begin{longtable}[r]{>{\ttfamily}l<{} p{12.6cm}}
		      \textsf{\textbf{Parameter}} & \textsf{\textbf{Values}}\\
		      \toprule
		      \endfirsthead
		      \textsf{\textbf{Parameter}} & \textsf{\textbf{Values}}\\
		      \toprule
		      \endhead
		      \bottomrule
		      \endfoot
		      \bottomrule
		      \endlastfoot
           type &	\vspace*{-1.28\baselineskip}
             \begin{itemize*}
               \item\mcode{'Pearson'}, (the default) computes Pearson's linear correlation coefficient
               \item\mcode{'Kendall'}, computes Kendall's $\tau$
				    \item\mcode{'Spearman'}, computes Spearman's $\rho$
               \item a matrix or \myfints{} object serveing as weights, calculating weighted correlation.
                if matrix, must be compatible with size;
                if \myfints{}, must be aligned with \mcode{iftsA} and \mcode{iftsB}.
                Parameter \mcode{rows} is not applicable for weighted correlation (if provided, ignored.)\newline
                \vspace*{-2\baselineskip}
             \end{itemize*}\\
           \midrule
           rows & \vspace*{-1.28\baselineskip}
             \begin{itemize*}
              \item \mcode{'all'} (the default) uses all rows regardless of missing values (\mcode{NaN}s)
              \item \mcode{'complete'} uses only rows with no missing values
              \item \mcode{'pairwise'} computes $\rho(i,j)$ using rows with no missing 
                    values in column $i$ or $j$\newline
              \vspace*{-2\baselineskip}
             \end{itemize*}\\
       \end{longtable}
   \end{argdesc}
%
   \mcode{csrankcorr} actually is equivelent to
   \begin{lstlisting}[numbers=none]
   ofts = cscorr(iftsA, iftsB, 'type','spearman','rows','complete');
   \end{lstlisting}
   and \mcode{cscorrwt} is equivelent to
   \footnote{Weighted mean is $m[x;w] = (\sum_i w_i*x_i)/(\sum_i w_i)$,
				  weighted covariance $c[x,y;w] = (\sum_i w_i*(x_i - m[x;w])*(y_i - m[y;w]))/(\sum_i w_i)$,
				  and then weighted correlation $r[x,y;w] = c[x,y;w]/\sqrt{c[x,x;w]*c[y,y;w]}$.}

   \begin{lstlisting}[numbers=none]
   ofts = cscorr(iftsA, iftsB, 'type', iftsWt);
   \end{lstlisting}
   Both of them are shortcut forms of \mcode{cscorr}.

\outarg
   \begin{argdesc}
	 \item [ofts] a \myfints{} object of cross section correlation.
   \end{argdesc}

\funentry{cscov}
   Corss-sectional covarance. 

\usage
   \begin{lstlisting}[numbers=none]
   ofts=cscov(iftsA, iftsB)               % NaN will be counted
   ofts=cscov(iftsA, iftsB, 'ignorenan')  % NaN will be excluded 
   \end{lstlisting}

\funentry{csmax, csmean, csmedian, csstd, cssum, csnorm}
   Cross-sectional statistics for a \myfints{} object.

\usage
   \begin{lstlisting}[numbers=none]
   ofts = csmax(fts)
   ofts = csmean(fts)
   ofts = cssum(fts)
   ofts = csmedian(fts)
   ofts = csstd(fts)
   ofts = csnorm(fts)
   \end{lstlisting}

Note all these functions ignore \mcode{NaN} values.

What if you want to a statistics not listed above? 
Using \mcode{uniftsfun}.
For example, suppose you want to calculate cross-sectional kurtosis
for a \myfints{}, simply do this
\begin{lstlisting}[numbers=none]
   ofts = uniftsfun(fts, @(x) kurtosis(x,1,2))
\end{lstlisting}
See also \matlab's help on \mcode{kurtosis}
for meanings of other two parameters.
\medskip

\funentry{ftsmovavg, ftsmovstd, ftsmovsum}
   Calculate the moving average/standard deviation/summation of a time series.
   
\usage
   \begin{lstlisting}[numbers=none]
    newFts = ftsmovavg(oldFts, window, ignoreNaN)
    newFts = ftsmovstd(oldFts, window, ignoreNaN)
    newFts = ftsmovsum(oldFts, window, ignoreNaN)
   \end{lstlisting}
\where
   \mcode{window} is the number of rows included in every calculation from current time point 
   backward to previous time point (including current time point).
   \mcode{NaN}s will be ignored in calculation if \mcode{ignoreNaN = 1}.
   
If you want to perform functions other than average/standard deviation/summation,
use \mcode{ftsmovfun}.


\subsection{Third-Dimensional Statistics}
\funentry{ftsnanmean, ftsnanstd, ftsnansum}

\usage
   \begin{lstlisting}[numbers=none]
    ofts = ftsnanmean(ifts1, ifts2, ..., iftsN)
    ofts = ftsnanstd(ifts1, ifts2, ..., iftsN)
    ofts = ftsnansum(ifts1, ifts2, ..., iftsN)
   \end{lstlisting}
\inarg
   \begin{argdesc}
   \item[ifts1, ifts2, ..., iftsN] \myfints{} objects on which specific statistics
        will be performed.
        They must be aligned.
   \end{argdesc}

\outarg
   \begin{argdesc}
   \item[ofts] the result of type \myfints{}, compatible with \mcode{ifts1},\ldots,\mcode{iftsN}, etc. 
   \end{argdesc}

  
\subsection{Performing Specified Functions on \texttt{myfints}}

\funentry{uniftsfun}
    Core function performing actions when only one myfints object
    involved.
    Acutally the \mcode{csmean}, \mcode{csmax}, etc. just a direct call to
    this function.

\usage
   \begin{lstlisting}[numbers=none]
    ofts = uniftsfun(ifts, fun_handle)
    ofts = uniftsfun(ifts, fun_handle, fieldname)
   \end{lstlisting}

\where
   \begin{argdesc}
    \item[ifts] \myfints{} object to be operated on
    \item[fun\_handle] operation to be performed on \mcode{ifts}.
                 \mcode{fun\_handle} should be a function of \par
                 \hspace{1.0cm}\mcode{function out\_matrix = fun(in\_matrix)} \par
                 where the dimension of \mcode{out\_matrix} 
                 can be different from that of \mcode{in\_matrix};
                 see explanation for \mcode{fieldname} below in this case.                 
    \item[fieldname] Specifies what field names should be in the returned
                 \myfints{} object \mcode{ofts}.
                 \mcode{fieldname} should be a char vector (representing a string)
                 or a cell vector of strings. 
                 The number of field names provided should be 1 (a char
                 vector or a one-element cell) or matching the result
                 returned by \mcode{fun\_handle}. 

                 In case of one field name provided and returning 
                 multiple fields (columns) by \mcode{fun\_handle}, 
                 the field name will be appended by a counter.
                 If number of fields names beyond 1 and not matching the 
                 result of \mcode{fun\_handle} (not equal the number of columns
                 returned by \mcode{fun\_handle}), 
                 an error will occur and break the
                 program execution.

                 By \emph{DEFAULT}, 
                 the \mcode{fieldname} in \mcode{ifts} will be used.
                 This is fine if input and output have the same columns.
                 Otherwise,
                 user should provide an appropriate \mcode{fieldname}.

                 For most cross-sectional operation,
                 user should provide a \mcode{fieldname}
                 since the returned usually 
                 has less than the original number of columns (mostly only
                 one column, such as mean, sum, etc. on column dimension)
                 For most time-series operation,
                 user don't need to provide \mcode{fieldname} since most of these
                 operation returned something having the same number of
                 columns as operand 
                 (still, consider \mcode{mean}, \mcode{sum} on row dimension).
 
                 Generally speaking, operation along a specified dimension
                 usually removes that dimension from the result. 
                 For instance, \mcode{mean(A,1)} will remove row(1st) dimension, 
                 and \mcode{mean(A,2)} remove the column(2nd) dimension.
          \item[ofts] result returned. 
                 Usually, it's still a \myfints{} object
                 having the similar structure as \mcode{ifts}
                 (see explanation about \mcode{fieldname} above). 
                 HOWEVER, in case of 
                 \textit{the result returned by \mcode{fun\_handle} has different number
                 of rows as that of inputted \mcode{ifts}},
                    \begin{itemize*}
                    \item if the returned row number is 1, 
                          \mcode{ofts} uses last date in \mcode{ifts} as its date; 
                    \item  otherwise, return directly the result (type of matrix)
                           instead of \myfints{} object.
                    \end{itemize*}
   \end{argdesc}

\noindent Examples:
  \begin{lstlisting}
        function ofts = csmedian(fts)
            ofts = uniftsfun(fts, @(x)nanmedian(x,2), 'csmedian');
        end
        
        function ofts = cssum(fts)
            ofts = uniftsfun(fts, @(x)nansum(x,2), 'cssum');
        end
  \end{lstlisting}

\funentry{biftsfun}
    Core function performing actions on at least two \myfints{} objects.
    \footnote{
    More \myfints{} objects possible, all depends on \mcode{fun\_handle}.
    \mcode{csqtspread} is such an example where at most 4 \myfints{}
    object involved: one for signal, one for return, one for GICS, one for weights.}

\usage
   \begin{lstlisting}[numbers=none]
    ofts = biftsfun(lhs, rhs, fun_handle)
    ofts = biftsfun(lhs, rhs, fun_handle, fieldname)
   \end{lstlisting}
\where
    \begin{argdesc}
    \item[lhs, rhs] operands (left and right in case of \mcode{fun\_handle} being 
                 bi-operand operator) on which \mcode{fun\_handle} to be performed.
                 One of them surely is \myfints{} object (otherwise \matlab{}
                 WON'T call this function). The other can be anything
                 \mcode{fun\_handle} allowed (e.g., a scalar, matrix, or \myfints{}
                 object, etc.)
                 \mcode{lhs} and \mcode{rhs} must be compatible in the sense of:
                   \begin{itemize*}
                   \item if both are \myfints{} objects, \mcode{iscompatible} returns \mcode{true}
                   \item if only one of them is of type \myfints{}, the matrix
                     form of them are allowed by \mcode{fun\_handle}
                   \end{itemize*}
    \item[fun\_handle] operation to be performed on \mcode{lhs} and \mcode{rhs}.
                The function pointed to by \mcode{fun\_handle} should be of form \par
                \hspace{1.0cm} \mcode{function out\_matrix = fun(in\_matrixA, in\_matrixB)}\par
                where \mcode{in\_matrixA} and \mcode{in\_matrixB} have the same dimension,
                and \mcode{out\_matrix} can be different of dimension from them.
                See explanation for \mcode{fieldname} below.
    \item[fieldname] Specifies what field names should be in the returned
                 \myfints{} object \mcode{ofts}.
                 \mcode{fieldname} should be a char vector (representing a string)
                 or a cell vector of strings. 
                 The number of field names provided should be 1 (a char
                 vector or a one-element cell) or matching the result
                 returned by \mcode{fun\_handle}. 

                 In case of one field name provided and returning 
                 multiple fields (columns) by \mcode{fun\_handle}, 
                 the field name will be appended by a counter.
                 If number of fields names beyond 1 and not matching the 
                 result of \mcode{fun\_handle} (not equal the number of columns
                 returned by \mcode{fun\_handle}), 
                 an error will occur and break the
                 program execution.

                 By \emph{DEFAULT}, 
                 the \mcode{fieldname} in \mcode{ifts} will be used.
                 This is fine if input and output have the same columns.
                 Otherwise,
                 user should provide an appropriate \mcode{fieldname}.
          \item[ofts] result returned. 
                 Usually, it's still a \myfints{} object
                 having the similar structure as \mcode{ifts}
                 (see explanation about \mcode{fieldname} above). 
                 HOWEVER, in case of 
                 \textit{the result returned by \mcode{fun\_handle} has different number
                 of rows as that of inputted \mcode{ifts}},
                    \begin{itemize*}
                    \item if the returned row number is 1, 
                          \mcode{ofts} uses last date in \mcode{ifts} as its date; 
                    \item  otherwise, return directly the result (type of matrix)
                           instead of \myfints{} object.
                    \end{itemize*}
      \end{argdesc}


\funentry{multiftsfun}
   Perform an element-by-element operation on multiple \myfints{} objects.

\usage
   \begin{lstlisting}[numbers=none]
   ofts = multiftsfun(ifts1, ifts2, ..., iftsN, fun_handle)
   \end{lstlisting}

\inarg
   \begin{argdesc}
   \item[ifts1, ifts2, ..., iftsN] \myfints{} objects on which operation specified by
        \mcode{fun\_handle} will be performed.
        They must be aligned.
        In the following, we suppose each of them is of size $T\times N$.
   \item[fun\_handle] actions to be performed on these \myfints{}.
        The function pointed by \mcode{fun\_handle} should be of from\par
        \hspace{1.0cm}\mcode{function out\_2D\_matrix = fun(in\_3D\_array)}\par
        where \mcode{in\_3D\_array} is of size $T\times N\times K$ and $K$ 
        corresponds to the number of \myfints{} inputted as parameters of
        \mcode{multiftsfun}, 
        and \mcode{out\_2D\_matrix} is of $T\times N$.
        For example, if you want to perform a element-by-element summation among
        a group of \myfints{}, you may set \mcode{fun\_handle} as \par
        \hspace{1.0cm}\mcode{fun\_handle = @(x) sum(x,3)}\par
        and keep in mind that \mcode{x} is a 3D matrix. 
   \end{argdesc}
\outarg
   \begin{argdesc}
   \item[ofts] the result of type \myfints{}, compatible with \mcode{ifts1},\ldots,\mcode{iftsN}, etc. 
   \end{argdesc}

\noindent Example:
\begin{lstlisting}
function sfts = ftsnansum(varargin)
    sfts = multiftsfun(varargin{:}, @(x)nansum(x,3));
end
\end{lstlisting}

\funentry{ftsmovfun}
   Calculate the moving statistics (e.g., moving average) of a time series.

\usage
   \begin{lstlisting}[numbers=none]
   ofts = ftsmovfun(ifts, window, fun_handle)
   ofts = ftsmovfun(ifts, window, fun_handle, fieldname)
   \end{lstlisting}
\inarg
   \begin{argdesc}
   \item[ifts] (\myfints{} object) 
   \item[window] size of moving window.
         If windows is \mcode{inf}, then it's the expanding window.
   \item[fun\_handle] operation to be performed in moving manner.
        It should be like \par
        \hspace{1.0cm}\mcode{function out\_matrix = fun(in\_matrix)}\par
        and keeping in mind that \mcode{in\_matrix} is the current moving part matrix.
        The number of columns of returned \mcode{out\_matrix}
        can be different from that of \mcode{in\_matrix}.
        The row numbers, however, must be 1
        (otherwise, it'll be confused where's the moving comes from).
        See explanation for \mcode{fieldname} below.
   \item[fieldname] Specifies what field names should be in the returned
                 \myfints{} object \mcode{ofts}.
                 \mcode{fieldname} should be a char vector (representing a string)
                 or a cell vector of strings. 
                 The number of field names provided should be 1 (a char
                 vector or a one-element cell) or matching the result
                 returned by \mcode{fun\_handle}. 

                 In case of one field name provided and returning 
                 multiple fields (columns) by \mcode{fun\_handle}, 
                 the field name will be appended by a counter.
                 If number of fields names beyond 1 and not matching the 
                 result of \mcode{fun\_handle} (not equal the number of columns
                 returned by \mcode{fun\_handle}), 
                 an error will occur and break the
                 program execution.

                 By \emph{DEFAULT}, 
                 the \mcode{fieldname} in \mcode{ifts} will be used.
                 This is fine if input and output have the same columns.
                 Otherwise,
                 user should provide an appropriate \mcode{fieldname}.
   \end{argdesc}
\outarg
   \begin{argdesc}
   \item [ofts] the resulting \myfints{} object.
   \end{argdesc}

\noindent Examples:
   \begin{lstlisting}
function newFts = ftsmovsum(oldFts, window, ignoreNaN)
if nargin < 3
    ignoreNaN = 1;
end

if ignoreNaN
    fun = @(x) nansum(x,1);
else
    fun = @(x) sum(x,1);
end

newFts = ftsmovfun(oldFts, window, fun);
   \end{lstlisting}

In \mcode{ftsmovsum}, if \mcode{window == inf}, then it actually calculates
cumulated sum.
 
\funentry{bsxfun\label{sec:bsxfun}}
  Apply element-by-element binary operation to two \mcode{myfints} or
  a \mcode{myfints} and an array with singleton expansion enabled.

\usage
   \begin{lstlisting}[numbers=none]
   C = bsxfun(fun,A,B)
   \end{lstlisting}

\where
   \begin{argdesc}
   \item[A,B] operands. at the least one of them is a \myfints{} object 
        (otherwise \matlab{} won't be here); the other can be an ordinary 
        \matlab{} matrix or \myfints{} object.
   \item[fun] a function handle, and can either be an \mcode{M}-file function or 
        one of the following built-in functions:
		   \begin{longtable}[c]{>{\ttfamily}l<{} p{8cm}}
		      \textsf{\textbf{Function}} & \textsf{\textbf{Operation}}\\
		      \toprule
		      \endfirsthead
		      \textsf{\textbf{Function}} & \textsf{\textbf{Operation}}\\
		      \toprule
		      \endhead
		      \bottomrule
		      \endfoot
		      \bottomrule
		      \endlastfoot
				@plus    & Plus\\
				@minus   & Minus\\
				@times   & Array multiply\\
				@rdivide & Right array divide\\
				@ldivide & Left array divide\\
				@power   & Array power\\
				@max     & Binary maximum\\
				@min     & Binary minimum\\
				@rem     & Remainder after division\\
				@mod     & Modulus after division\\
				@atan2   & Four quadrant inverse tangent\\
				@hypot   & Square root of sum of squares\\
				@eq      & Equal\\
				@ne      & Not equal\\
				@lt      & Less than\\
				@le      & Less than or equal to\\
				@gt      & Greater than\\
				@ge      & Greater than or equal to\\
				@and     & Element-wise logical AND\\
				@or      & Element-wise logical OR\\
				@xor     & Logical exclusive OR\\
  		   \end{longtable}
  		   
  		   If an \mcode{M}-file function is specified, 
  		   it must be able to accept either two column vectors of the same size, 
  		   or one column vector and one scalar, and return as output a column vector of the size as the input values.
   \end{argdesc}

   \mcode{bsxfun} applies an element-by-element binary operation to \mcode{A} and \mcode{B}, 
   with singleton expansion enabled, 
   The operation performed \emph{irrespective of the compatibility of \mcode{A} and \mcode{B}}
   when both of them are \myfints{} objects, so long as their dimensions agree.
   By ``agree'', we mean each dimension of \mcode{A} and \mcode{B} must either be equal to each other, 
   or equal to 1. 
   Whenever a dimension of \mcode{A} or \mcode{B} is singleton (equal to 1),
   the array is virtually replicated along the dimension to match the other array. 
   The array may be diminished if the corresponding dimension of the other array is 0.
   
   The result returned usually still is a \myfints{} object,
   with dates and field names match that of first suitable in \mcode{A} and \mcode{B} (in order of occurrence in argument list).
   If no matched dates found with the result, the result returned in matrix form.
   It will be an error no matched field names found with the result.
   (However, if the number of field names found is 1, the name will be appended with a counter
   up to the number of columns of result as the result's field name).


\subsection{Field Operations}
This section lists function related to manipulating \myfints{}' fields.

\begin{lstlisting}[numbers=none]
    fnames = fieldnames(fts, 1)          % return field names
    fnames = fieldnames(fts, 0)          % return field names, incl. freq, dates, desc

    ret = getfield(fts, fids)            % => fts(:,flds)
    ret = getfield(fts, fids, dates)     % => fts(dates,flds)

    fts = setfield(fts,flds,val)         % => fts(:,flds) = val
    fts = setfield(fts,flds,dates,val)   % => fts(dates, flds) = val

    fts = extfield(fts, flds)            % => fts = fts(:,flds);
    fts = rmfield(fts, rmflds)           % => fts = fts(:, setdiff(fieldnames(fts,1),rmflds))

    fts = chfield(fts,oldnames,newnames) % change field names; newnames 1-to-1 oldnames

    isfld = isfield(fts, fldname)        % check if fldname is a field of fts

    fts = padfield(fts, flds)            % NOT => fts = fts(:,flds), padding new fields with NaNs
    fts = padfield(fts, flds, padStuff)  % NOT => fts = fts(:,flds), padding new fields with padStuff
\end{lstlisting}

Note that \mcode{fts = padfield(fts, flds, padStuff)} is not equivalent to \mcode{fts = fts(:,flds)}; 
it allows flds not be existing fields of \mcode{fts}, in which case new fields will be added to fts 
with values being \mcode{padStuff}
which by default is \mcode{NaN}.

\subsection{Resampling}
\funentry{convertto, toannual, tosemi, toquarterly, tomonthly, toweekly, todaily}
\begin{lstlisting}[numbers=none]
   ofts = convertto(ifts, freq, 'argname', argvalue, ...)
	ofts = toannual(ifts, 'argname', argvalue, ...)
	ofts = tosemi(ifts, 'argname', argvalue, ...)
	ofts = toquarterly(ifts, 'argname', argvalue, ...)
	ofts = tomonthly(ifts, 'argname', argvalue, ...)
	ofts = toweekly(ifts, 'argname', argvalue, ...)
	ofts = todaily(ifts, 'argname', argvalue, ...)
\end{lstlisting}

\where
    \mcode{argname} and \mcode{argvalue} can be:
	      \begin{longtable}[r]{>{\ttfamily}l<{} c p{8cm} p{2.5cm}}
	  \textsf{\textbf{Parameter}} & \textsf{\textbf{Default}} & \textsf{\textbf{Meaning}} & \textsf{\textbf{Freqs Appl.}}\\
	  \toprule
	  \endfirsthead
	  \textsf{\textbf{Parameter}} & \textsf{\textbf{Default}} & \textsf{\textbf{Meaning}} & \textsf{\textbf{Freqs Appl.}}\\
	  \toprule
	  \endhead
	  \bottomrule
	  \endfoot
	  \bottomrule
	  \endlastfoot
      \mcode{'EOW'} & 0 & 0 - 6, Specifies the end-of-week day:\newline
					  0--Friday,  1--Saturday, 2--Sunday, 3--Monday,
                 4--Tuesday, 5--Wednesday, 6--Thursday
					& weekly(2) \\
      \mcode{'ED'} & 0 & The end of date (of month) in a period. 0 is the last day (or last business day),
                     1-31 is the specific day (if beyond end of month, adjust to end of month)
	               & monthly(3), quarterly(4), semiannual(5), annual(6) \\
      \mcode{'EM'} & 3 & Last month of the first period (e.g., quarter, year). 
	               All subsequent quarterly dates are based on this month
				   & quarterly(4), semiannual(5), annual(6) \\
	   \mcode{'Busdays'} & 0 & Either 0 or 1, indicating whether business days (non-weekend and non-holdidays) are counted.
                  If a sampling day is fallen in a non-business day, it will be adjusted to the next business day, 
                  unless there's no business day in the current frequency period (e.g., current month), 
                  in which case prior business day will be used.
	               & All frequencies \\
      \mcode{'Holidays'} & NYSE & A vector specifying holidays & All frequencies \\
      \mcode{'Weekend'}  & \footnotesize[1 0 0 0 0 0 1] & Specify which day is weekend in order of [Sun Mon Tue Wed Thu Fri Sat] & All frequencies \\
      \end{longtable}

	
	There is an additional named argument.
	 Following a char-type string indicator parameter \mcode{'CalcMethod'} (case-insensitive),
 specifies which method is used to calculate resampling sample data.
 Methods could be one of
 \begin{itemize}
 \item \mcode{'CumSum'}
 Returns the cumulative sum of the values between each sampling period (e.g., week, month, etc.)
 Data for missing dates are given the value 0.
 \item \mcode{'Exact'}
 Returns the exact value at the end-of-sampling-period date. No data manipulation occurs.
 \item \mcode{'Nearest'}
 (\emph{default}) Returns the values located at the end-of-sampling-period date. 
 If there is missing data, Nearest returns the nearest data point preceding the end-of-sampling-period date. 
 \item \mcode{'SimpAvg'}
 Returns an averaged value over each sampling period 
 that only takes into account dates with data (non NaN) within each quarter.
 \end{itemize}

	

\subsection{List of Other Functions}
Some are new, most have the same signatures as those of \matlab's financial toolbox.
{\ttfamily
    \begin{itemize}
	\item not($\sim$) \textnormal{and} power($.$\^{})
	\item isnan, isinf
	\item log, log10, log2
	\item exp
	\item abs, diff, cumsum\\
	      \textnormal{All function above (from \mcode{not}) return another \myfints{} object.}
	\item max, min, mean, std \\
	      \textnormal{Functions below return a structure, one evidence of inconsistence in \matlab's fints.}
	\item tsmovavg, macd \\
	      \textnormal{These two functions also implemented on ordinary matrix.}
	\item var, cov, corrcoef \\
	      \textnormal{These three functions return a ordinary matrix.}
   \item nanmax, nanmin, nanmean, nanmedian, nansum, nanstd, nanvar, nancov \\
	      \textnormal{All these nan-functions return a normal matrix, 
		  inconsistent with their none-nan counterparts which many return a weird structure.
		  Note their matrix version implemented in \matlab's stat toolbox.}
   \item plot 
	      \textnormal{Now directly call \mcode{tsplot}.}
	\item iscompatible, isempty, isequal, issorted, size, length, fts2mat
   \item \textnormal{Lag and Lead Operators the same as before:}
   \begin{lstlisting}[numbers=none]
    fts = lagts(fts, nperiod, padmode)
    fts = leadts(fts, nperiod, padmode)
   \end{lstlisting}
   \item \textnormal{Concatenating Operators:}
   \begin{lstlisting}[numbers=none]
    fts = horzcat(fts1, fts2, ..., ftsN)   % => [fts1, fts2, ..., ftsN]
    fts = vertcat(fts1, fts2, ..., ftsN)   % => [fts1; fts2; ...; ftsN]
	\end{lstlisting}
		\textnormal{\mcode{vertcat} has been enhanced to allow date overlapping in \myfints{} objects to be mergerd,
		in which case the first occurance data used in the result.}
	\end{itemize}
}
