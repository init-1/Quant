\section{Mutliple Dimensional Time Series: xts\label{sec:xts}}

Matrix of dimensions beyond 2 in \matlab{} is a natural extension of 2-D matrix.
So is out multiple dimension financial times series class: \mcode{xts}.
The usage is consistent with \matlab{}'s matrix operations and common sense.

\subsection{Common Functions with slightly different usage}

For easy comparation, we list them in tabel \ref{tab:xts}.

\subsection{Common Functions}

Function which have the same usage as myfints are listed in table \ref{tab:commonxts}.

\begin{landscape}
\footnotesize
\begin{longtable}[c]{>{\ttfamily}l<{} p{7cm} p{8cm} p{6cm}}
  \caption{Comparison between \mcode{xts} and \myfints{}\label{tab:xts}}\\
  \textsf{\textbf{method name}} & \textsf{\textbf{myfints}} & \textsf{\textbf{xts}} & \textsf{\textbf{notes}} \\
  \toprule
  \endfirsthead
  \textsf{\textbf{method name}} & \textsf{\textbf{myfints}} & \textsf{\textbf{xts}} & \textsf{\textbf{notes}} \\
  \toprule
  \endhead
  \bottomrule
  \endfoot
  \bottomrule
  \endlastfoot
  \rowcolor{cone}
  constructor & 
            fts = myfints\newline
            fts = myfints(dates, data)\newline
            fts = myfints(dates, data, fieldnames)\newline
            fts = myfints(dates, data, fieldnames, freq)\newline
            fts = myfints(dates, data, fieldnames, freq, desc)\newline
            fts = myfints(dates, data, fieldnames, freq, desc, unit) 
            &
            xts = xts\newline
            xts = xts(dates, data)\newline
            xts = xts(dates, data, fieldnames1, ..., fieldnamesN)\newline
            xts = xts(dates, data, fieldnames1, ..., fieldnamesN, freq)\newline
            xts = xts(dates, data, fieldnames1, ..., fieldnamesN, freq, desc)\newline
            xts = xts(dates, data, fieldnames1, ..., fieldnamesN, freq, desc, unit)
            &
            $N$ is called field dimension. \mcode{xts} is a $N+1$ dimension \mcode{xts}. \\
\rowcolor{ctwo}
 alignfields &
           [ofts1, ofts2, ...] = aligndata(ifts1, ifts2, ...)\newline
           [ofts1, ofts2, ...] = aligndata(ifts1, ifts2, ..., mode)\newline
           \newline
           \newline
           &
           [oxts1, oxts2, ...] = aligndata(ixts1, ixts2, ...)\newline
           [oxts1, oxts2, ...] = aligndata(ixts1, ixts2, ..., mode)\newline
           [oxts1, oxts2, ...] = aligndata(ixts1, ixts2, ..., flddim)\newline
           [oxts1, oxts2, ...] = aligndata(ixts1, ixts2, ..., mode, flddim)\newline
           [oxts1, oxts2, ...] = aligndata(ixts1, ixts2, ..., flddim, mode)
           &
           Aligning specified field dimension.
           if \mcode{flddim} == 0 (default), all field dimensions will be aligned.
           Input arg \mcode{xts} objects can be different dimensional so long as they all have
           the specified field dimension.\\
\rowcolor{cone}
 aligndata &
           \multicolumn{2}{p{15cm}}{
           [ofts1, ofts2, ...] = aligndata(ifts1, ifts2, ...)\newline
           [ofts1, ofts2, ...] = aligndata(ifts1, ifts2, ..., mode)\newline
           [ofts1, ofts2, ...] = aligndata(ifts1, ifts2, ..., aligningdates)\newline
           [ofts1, ofts2, ...] = aligndata(ifts1, ifts2, ..., mode, aligningdates)\newline
           [ofts1, ofts2, ...] = aligndata(ifts1, ifts2, ..., aligningdates, mode)\newline
           [ofts1, ofts2, ...] = aligndata(..., 'CalcMethod', calcMethod)
           }
           & Align data along all dimensions (time and all field dimensions).
           So all \mcode{xts} objects must have the same dimension.\\
\rowcolor{ctwo}
isaligneddata&
          tf = isaligneddata(fts1, ..., ftsN) &
          tf = isaligneddata(xts1, ..., xtsN)\newline
          tf = isaligneddata(xts1, ..., xtsN, flddim) &
          \mcode{flddim} by default is 0, means check all field dimensions as well as time dim.\\
\rowcolor{cone}
isalignedfields&
          tf = isalignedfields(fts1, ..., ftsN) &
          tf = isalignedfields(xts1, ..., xtsN) \newline
          tf = isalignedfields(xts1, ..., xtsN, flddim) &
         \mcode{flddim} is 0 by default.\\
\rowcolor{ctwo}
fieldnames &
          fn = fieldnames(fts)\newline
          fn = fieldnames(fts, 1)&
          fn = fieldnames(xts, 0, flddim)\newline
          fn = fieldnames(xts, 1, flddim)& 
          \mcode{flddim} is 1 by default.\\
\rowcolor{cone}
getfield &
         fval = getfield(fts, fields) \hspace{1cm} (fields can be multiple)\newline
         fval = getfield(fts, fields, dates) \hspace{0.3cm} (fields can be multiple)&
         fval = getfield(xts, field)  \hspace{1.6cm}(field must be char)\newline
         fval = getfield(xts, 'dates', dateidx) \hspace{.3cm}(equiv. xts.dates(idx))&
         Unless accessing \mcode{xts.freq}, \mcode{xts.desc}, \mcode{xts.unit}, or \mcode{xts.dates},
         you are recommended to use \mcode{xts(timeidx, fidx1, ..., fidxN)} to get part of \mcode{xts}.\\
\rowcolor{ctwo}
setfield &
        fts = setfield(fts, fields, value)\newline
        fts = setfield(fts, fields, dates, value) &
        xts = setfield(xts, field, value)\newline
        xts = setfield(xts, 'dates', dateidx, value) &
        Similiar to \mcode{getfield} except a value need to be provided.\\
\rowcolor{cone}
extfield &
       fts = extfield(fts, fldnames) &
       xts = extfield(xts, fldnames, flddim = 1) &\\
\rowcolor{ctwo}
rmfield &
       fts = rmfield(fts, fldnames) &
       xts = rmfield(xts, fldnames, flddim = 1) &\\
\rowcolor{cone}
padfield &
       fts = padfield(fts, fldnames, padstuff = NaN) &
       xts = rmfield(xts, fldnames, padstudd = NaN, flddim = 1) &\\
\rowcolor{ctwo}
chfield &
       fts = chfield(fts, oldfldnames, newfldnames) &
       xts = chfield(xts, oldfldnames, newfldnames, flddim = 1)&\\
\rowcolor{cone}
isfield &
       tf = isfield(fts, fldname) &
       tf = isfield(xts, fldname, flddim = 1) &\\
\rowcolor{ctwo}
fts2mat &
       mat = fts2mat(fts) \newline
       mat = fts2mat(fts, datesflag) \newline
       mat = fts2mat(fts, fldnames) \newline
       mat = fts2mat(fts, datesflag, fldnames) &
       mat = fts2mat(xts) & 
       \mcode{xts} only allow obtaining whole data, excluding dates.\\
\rowcolor{cone}
cat &
      fts = cat(1, fts1, ..., ftsN) or fts = [fts1, ..., ftsN]\newline
      fts = cat(2, fts1, ..., ftsN) or fts = [fts1; ...; ftsN] &
      xts = cat(1, xts1, ..., xtsN) or xts = [xts1, ..., xtsN]\newline
      xts = cat(2, xts1, ..., xtsN) or xts = [xts1; ...; xtsN]\newline
      xts = cat(dim, xts1, ..., xtsN) &
      Vertical cat (dim==1, or in form of [fts1;...; ftsN]) will order data by dates in final result.\\
\rowcolor{ctwo}
uniftsfun\newline
biftsfun &
    xts = uniftsfun(xts, fun, fields, dates) \newline
    xts = biftsfun(lxts, rxts, fields, dates) &
    lxts and rxts must be aligned in all dimensions.
    Now fields can be a cell vector with each element is a cell of strings representing corresponding field dimension's fieldnames.
    If you want use the original xts' fieldnames, leave corresponding posn an empty thing 
    (of course, this dimension in result should be matched to that in original xts arguments.
\end{longtable}
\end{landscape}

\begin{landscape}
\footnotesize
\begin{longtable}{l p{8cm} p{8cm}}
  \caption{Same usage functions\label{tab:commonxts}}\\
  \textsf{\textbf{method name}} & \textsf{\textbf{usage}} & \textsf{\textbf{notes}} \\
  \toprule
  \endfirsthead
  \textsf{\textbf{method name}} & \textsf{\textbf{usage}} & \textsf{\textbf{notes}} \\
  \toprule
  \endhead
  \bottomrule
  \endfoot
  \bottomrule
  \endlastfoot
  \rowcolor{cone}
  aligndates &
           [ofts1, ofts2, ...] = aligndates(ifts1, ifts2, ...)\newline
           [ofts1, ofts2, ...] = aligndates(ifts1, ifts2, ..., aligningdates)\newline
           [ofts1, ofts2, ...] = aligndates(..., 'CalcMethod', calcMethod)
           &
           Resampling along the time dimension.
           \mcode{'CalcMethod'} can be \mcode{'cumsum'}, \mcode{'exact'}, \mcode{'nearest'}(default),
           and \mcode{'simavg'}.
           Input arguments of \mcode{xts} can be different in number of dimensions.\\
 \rowcolor{ctwo}
 isaligneddates& tf = isaligneddates(fts1, ..., ftsN) & \\
 \rowcolor{cone}
 isempty & tf = isempty(xts) &\\
 \rowcolor{ctwo}
 isequal & tf = isequal(fts1, ..., ftsN) & \\
 \rowcolor{cone}
 size    & sz = size(xts) \newline 
           sz = size(xts, dim) \newline
           [sz1, sz2,...,szN] = size(xts) &\\
 \rowcolor{ctwo}
 length & len = length(xts) & Equivalent to size(xts,1) \\
 \rowcolor{cone}
 lagts \newline
 leadts &
        xts = lagts(xts, nperiod, padmode = 0)\newline
        xts = leadts(fts, nperiod, padmode = 0) 
        & padmode default value takes 0 for compatible with \matlab{} fints.\\
 \rowcolor{ctwo}
 Relation Ops & \texttt{>=, >, <, <=, ==, $\sim$=}, $sim$ & 
       One operand should be xts, the other either be a scalar, dimensional-match matrix, or aligned xts\\
 \rowcolor{cone}
 uniary Ops & -xts, +xts & $-$: flip sign of xts, $+$ has no effect \\
 \rowcolor{ctwo}
 Binary Ops & \texttt{+, -, *, /, .*, ./, \^{}} & Note that \texttt{*} and \texttt{/} are equivalent to \texttt{.*} and \texttt{./} \\
 \rowcolor{cone}
 \multicolumn{2}{l}{nanmax, nanmin, nanmean, nanmedian, nansum, nanstd, nanvar, var} & 
        Same as \matlab{}'s corresponding functions; all return ordinary matrix. For Compatible with \matlab{}'s fints\\
 \rowcolor{ctwo}
 \multicolumn{2}{l}{isnan, isinf, exp, log, log10, log2, abs} & Element-level ops; all return another xts \\
 \rowcolor{cone}
 cumsum, diff & oxts = cumsum(ixts), oxts = diff(ixts) & Ops along time dimension. Return another xts. Returned from diff has one less row\\
 \rowcolor{ctwo}
 \multicolumn{2}{l}{convertto, toannual, todaily, tomonthly, toquarterly, tosemi, toweekly} & Resampling along time dimension\\
 \rowcolor{cone}
 backfill\newline
 forwardfill &
         backfill(o, lookbackperiod, mode) \newline
         forwardfill(o, lookfwdperiod, mode) &
         Now when mode is 'entry', it fill along time dimension for all other dimensions.
         If it is 'row', filling will use data without NaNs in all dimensions.\\
\rowcolor{ctwo}
  multiftsfun & oxts = multiftsfun(xts1, ..., xtsN, fun) & Perform fun(xts1,...,xtsN); all xts must be aligned in all dimensions\\
\rowcolor{cone}
  bsxfun & oxts = bxsfun(fun, lxts, rxts) & Same usage as for 2-D. Note now singleton extension expanded to all dimensions. 
        First fit (in terms of dimension) field used in result. See \ref{sec:bsxfun}.
\end{longtable} 
\end{landscape}

\subsection{Functions only applicable for myfints}
\begin{itemize*}
\item  \mcode{nancov corrcoef cov}
\item  \mcode{max min mean std}: It's wield that these 4 return a structure in \matlab's fints
\item  All \mcode{csXXXX} functions
\item  All \mcode{ftsmovXXX} functions
\item  \mcode{ftsnanmean, ftsnanstd, ftsnansum, ftswgtmean}
\item  All \mcode{XXXXif} function which we don't recommend to use even for \mcode{myfints}
\item  Backtest functions: \mcode{neutralize, normalize, optimize, tsplot, plot} 
\item  \emph{Depreciated} functions: \mcode{datemismatch, fieldmismatch, isidenticaldates, isidenticalfields}
\item  Functions for compatibility with \matlab{}'s fints: \mcode{tsmovavg, macd}
\end{itemize*}

\subsection{Indexing}

Suppose we have a 4-D xts: first dimension is time, second for security ids, third for factors, and forth for long or short.
Further suppose size of time dim is 10, size of security dim is 10, size of factor dimension is 5, for factors {'fac1', 'fac2', 'fac3', 'fac4', 'fac5'}, and size of forth dimension is 2 for {'short', 'long'}. Our 4-D xts variable is x with size of $10\times10\times5\times2$.
Following examples show you how to indexing an xts.
\begin{lstlisting}[numbers=none]
  a = x(:,:,:, 'short');              % return a 3-D xts since last dimension is singleton.
  a = x(:,:,{'fac1','fac5'}, :);      % return a 4-D xts: 10x10x2x2
  a = x(:,:,'fac2', {'long'});        % return a myfints (10x10) since last two dimension are singletons. 
  a = x(:,:,'fac2', {'long', shor'}); % return 4-D xts: 10x10x1x2

  a = squeeze(x(:,:,'fac2', {'long', shor'}));
  % return a 3-D xts: 10x10x2, the factor dim disappeared since it's singleton

  a = x(:,5,:,'short');          
  % return a 3-D xts: 10x1x5, last dim is singleton and removed automatically

  a = squeeze(x(:,5,:,'short'));
  % return a myfints: 10x5, last dim removed automatically and security dim squeezed since it's singleton

  a = x(1,:,:,:);                    % return 4-D xts: 1x10x5x2
  a = x(1,:,'fac1','long');          % return myfints: 1x10, traling singleton dims removed 
  a = squeeze(x(1,:,'fac1','long'))  % same as above since myfints don't have squeeze ops
  a = squeeze(x(1,:,:,:));           % return a 10x5x2 matrix, since squeeze ops on an xts

  a = squeeze(x(1,:,'fac1',:))      
  % return a 3-D xts: 1x10x2, since squeeze ops on 1x10x1x2 xts which keeps time dim even it singleton
\end{lstlisting}

The rules are
\begin{enumerate}
   \item Traling singleon dimensions beyond 2 are always removed automatically.
   \item 2-D xts will automatically casted into a myfints.
   \item Squeeze() has no effect on a myfints.
   \item If some dimensions but time are singleton in an xts, squeeze() will remove them.
   \item If the only singleton dimension is time in an xts, squeeze() will remove time dimension and return a plain matrix. 
\end{enumerate}

\subsection{Packing Matrix into \mcode{xts}}
\funentry{mat2xts}
   Convert data from \matlab's vectors into an \mcode{xts} object.

\usage
   \begin{lstlisting}[numbers=none]
   xts = mat2xts(date, val, fields1, ..., fieldN)
   \end{lstlisting}

   Convert vectors to an (N+1) xts.

\inarg
   \begin{argdesc}
   \item[date, val] vectors represent dates and corresponding values, receptively.
   \item[field1,...,fieldN]  vector represents individual field dimensions (e.g., securities ids).
   \end{argdesc}

   Note that the sizes of \mcode{date}, \mcode{val}, \mcode{fields1},..., and \mcode{fieldsN}
   must be agree with each other.
   The orders of these vectors are irrelevant (ordering will be done inside \mcode{mat2xts}).
   
\outarg
   \begin{argdesc}
   \item[xts] an $N+1$ dimentional \mcode{xts}.
              The number of periods (time points) is determined by number of unique values in \mcode{date};
              other dimension's size is determined by unique values in corresponding \mcode{fields1}.
   \end{argdesc}


