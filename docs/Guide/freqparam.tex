      \begin{longtable}[r]{>{\ttfamily}l<{} c p{8cm} p{2.5cm}}
	  \textsf{\textbf{Parameter}} & \textsf{\textbf{Default}} & \textsf{\textbf{Meaning}} & \textsf{\textbf{Freqs Appl.}}\\
	  \toprule
	  \endfirsthead
	  \textsf{\textbf{Parameter}} & \textsf{\textbf{Default}} & \textsf{\textbf{Meaning}} & \textsf{\textbf{Freqs Appl.}}\\
	  \toprule
	  \endhead
	  \bottomrule
	  \endfoot
	  \bottomrule
	  \endlastfoot
      \mcode{'EOW'} & 0 & 0 - 6, Specifies the end-of-week day:\newline
					  0--Friday,  1--Saturday, 2--Sunday, 3--Monday,
                 4--Tuesday, 5--Wednesday, 6--Thursday
					& weekly(2) \\
      \mcode{'ED'} & 0 & The end of date (of month) in a period. 0 is the last day (or last business day),
                     1-31 is the specific day (if beyond end of month, adjust to end of month)
	               & monthly(3), quarterly(4), semiannual(5), annual(6) \\
      \mcode{'EM'} & 3 & Last month of the first period (e.g., quarter, year). 
	               All subsequent quarterly dates are based on this month
				   & quarterly(4), semiannual(5), annual(6) \\
	   \mcode{'Busdays'} & 0 & Either 0 or 1, indicating whether business days (non-weekend and non-holdidays) are counted.
                  If a sampling day is fallen in a non-business day, it will be adjusted to the next business day, 
                  unless there's no business day in the current frequency period (e.g., current month), 
                  in which case prior business day will be used.
	               & All frequencies \\
      \mcode{'Holidays'} & NYSE & A vector specifying holidays & All frequencies \\
      \mcode{'Weekend'}  & \footnotesize[1 0 0 0 0 0 1] & Specify which day is weekend in order of [Sun Mon Tue Wed Thu Fri Sat] & All frequencies \\
      \end{longtable}
