\section{Introduction\label{sec:intro}}
\setcounter{page}{1}
\pagenumbering{arabic}

The \toolboxname{} defines a framework for constructing factors 
and provides a bunch of functions to support and facilitate the process,
including acquiring, sampling, aligning and manipulating basic data items
(financial like EPS and DPS and market-related like prices and shares)
upon which factors are built, 
and testing, visualizing and reporting tools.
These functions cover a wide range of functionalities and reach
every part of our quantitative infrastructure,
across from the very first stage of quantitative model development work 
to the last stage of generating the real trading data.
Most of our other tools, such as factor (linear and nonlinear) combination,
factor analysis and monitoring, performance attribution, 
model backtest and portfolio construction, are built on this toolbox.
Figure \ref{fig:whole} shows you the whole picture of it.

\begin{figure}[!hbp]
\centering
\caption{Structure of \toolboxname{}\label{fig:whole}}
\begin{tikzpicture}[node distance=0.1cm,auto]
  \tikzset{
    blknode/.style={rectangle, rounded corners, draw=black,
                   top color=white, bottom color=SkyBlue!50,
                   very thick, inner sep=5pt,
                   minimum width=3.5cm, minimum height=1cm,
                   column sep=5pt},
    subnode/.style={rectangle, rounded corners, draw=white,
                   top color=NavyBlue!10,bottom color=NavyBlue!60,
                   very thick, inner sep=2pt},
    myarrow/.style={->, >=latex', shorten >=1pt, thick},
    mylabel/.style={text width=7em, text centered},
  }

  \matrix [blknode] (base) {
	  \node[subnode] (xts) {xts}; &
	  \node[subnode] (db)  {DB}; \\
  };
 
  \matrix [blknode, below=of base] (myfints) {
	  \node[subnode] (my) {myfints}; &
	  \node[subnode] (myfints utilities) {myfints utilities};\\
  };

  \matrix [blknode, below=of myfints] (FacBase) {
     \node[subnode,minimum width=5.5cm,align=left] (local FacBase) {\textbf{FacBase\qquad\qquad}}; 
	  \node[subnode,minimum size=1mm, anchor=east,drop shadow] at ([xshift=-5pt]local FacBase.east) (DateBasis) {DateBasis}; &
     \node[subnode,minimum width=1.5cm] (Freq) {Freq}; \\
  };

  \def\step{1.25cm}
  \matrix [blknode, below=of FacBase, minimum height=0.1mm,row sep=-0.5cm] (Factors) {
     \node[minimum width=7.28cm, minimum height=0.1cm] (local Factors) {\textbf{Factors} (FactorLib)};\\
     \foreach \s in {1,...,3} 
     {
        \node[draw,color=white,circle,fill,top color=NavyBlue!15,bottom color=NavyBlue!65,minimum size=0.1mm,font=\footnotesize,circular drop shadow] 
        at ([xshift=\step*(\s)]local Factors.west){\color{black}fac\s};
     }
     \foreach \s in{1,2,3}
     {
        \shade[ball color=NavyBlue!50, opacity=1] ([xshift=3*\step+0.5cm+\s*12pt,yshift=-0.5cm]local Factors.west) circle (0.1);
     }
     \node[draw,color=white,circle,fill,top color=NavyBlue!15,bottom color=NavyBlue!65,minimum size=0.1mm,font=\footnotesize,circular drop shadow] 
        at ([xshift=\step*5]local Factors.west){\color{black}fac$n$};
     \\
  };

  \begin{scope}[every node/.style={font=\Huge\bfseries\sffamily},color=gray!70]
     \node at ([xshift=-20pt]base.west)    {1};
     \node at ([xshift=-20pt]myfints.west) {2};
     \node at ([xshift=-20pt]FacBase.west) {3};
     \node at ([xshift=-20pt]Factors.west) {4};
  \end{scope}

\end{tikzpicture}
\end{figure}

\begin{enumerate}
   \item The very first layer is quite fundamental, consisting of \mcode{xts} -- 
         multi-dimensional time series class representing the basic data form used in \matlab{},
         and \mcode{DB} -- interface class to Database for accessing data.
         \mcode{xts} can be regarded as labeled matrices with first dimension for time, others labeled with some meaningful names. 
         Labeled matrices greatly facilitates data sampling and aligning which are tedious chores and easily error-prone.

         \mcode{DB} isolates and wraps most of interactions with database.
         There are another group of functions (name started with \mcode{'Load'}) in \mcode{myfintsutility}
         which are actually another wrapping layer for \mcode{DB} class.
           
   \item This layer is \mcode{myfints} and related things in \mcode{myfintsutility} folder.
         \mcode{myfints} is 2-D \mcode{xts}, derived from \mcode{xts} (therefore having all methods from \mcode{xts}),
         plus things specific and more meaningful (e.g., \mcode{tsplot}) 
         (e.g., functions with names started with \mcode{'cs'}) in 2-D situation.
         Many useful tools are also placed here, like 
         \begin{itemize}
           \item \mcode{Options}, collecting operations related to processing named function arguments (name-value pairs),
           \item \mcode{grep}, regular search and/or replace string in files,
           \item \mcode{PDFDoc}, generating PDF documents with figures and tables,
           \item \mcode{TRACE}, logging facilities (to screen, files, and/or database),
         \end{itemize}
         to name a few.

   \item \mcode{FacBase} is derived from \mcode{myfints} and provides a framework on how to define a factor.
        Factors, associated with a security, is something that related to the future movements of the security prices;
        in other words, they have certain predictability.
        A typical factor, for example, is Book to Price, a ratio of company's book value to its share price,
        usually positively correlated to the company's stock price.
        \mcode{FacBase} take care of the details of such things as: 
        data processing (e.g., back filing missing data), 
        sampling (frequency internally used by items for calculation and frequency in factors returned to user),
        and aligning (both in time and other dimensions).
        It also provides methods for period calculation.

        Inside \mcode{FacBase}, there is a member of type \mcode{DateBasis} to deal with different (internal)
        frequencies, with the help from class \mcode{Freq}.

   \item This layer consists of classes of factors derived from \mcode{FacBase}.
        Since the base class \mcode{FacBase} deals with majority routine common things,
        concrete factor classes should only focus on the definition of factors.
        That is why code in factors usually is so neat.

        There also exists a class \mcode{Factory} in charge of populating factors to database.
\end{enumerate}


In this guide book,
we will first (section \ref{sec:steps}) present you the factor construction framework, detailing the steps how to create a factor.
Then move to discuss functions consisting of this toolbox by category.
Section \ref{sec:DBIO} briefs the database structure and related functions.
Since in most cases, we encounter and deal with 2-D matrices,
so section \ref{sec:myfints} discusses \mcode{myfints}, 
which is a 2-D version of \mcode{xts},
with many unique things and being most useful in quantitative research.
After that, \mcode{xts} will be introduced with comparison with \mcode{myfints}.







