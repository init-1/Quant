 Following a char-type string indicator parameter \mcode{'CalcMethod'} (case-insensitive),
 specifies which method is used to calculate resampling sample data.
 Methods could be one of
 \begin{itemize}
 \item \mcode{'CumSum'}
 Returns the cumulative sum of the values between each sampling period (e.g., week, month, etc.)
 Data for missing dates are given the value 0.
 \item \mcode{'Exact'}
 Returns the exact value at the end-of-sampling-period date. No data manipulation occurs.
 \item \mcode{'Nearest'}
 (\emph{default}) Returns the values located at the end-of-sampling-period date. 
 If there is missing data, Nearest returns the nearest data point preceding the end-of-sampling-period date. 
 \item \mcode{'SimpAvg'}
 Returns an averaged value over each sampling period 
 that only takes into account dates with data (non NaN) within each quarter.
 \end{itemize}
